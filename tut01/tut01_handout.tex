\documentclass[aspectratio=1610,onlymath, ngerman, handout]{beamer}
% \documentclass[aspectratio=1610,onlymath,handout]{beamer}
\usepackage[ngerman]{babel}
\usepackage[utf8]{inputenc}
\usepackage[T1]{fontenc}

\usetheme[nosectionnum,pagenum,sansmath, cd2018, nodin]{tud}

%\usefonttheme[onlymath]{serif}
\usepackage{opensans}
\usepackage{stmaryrd}
\usepackage[normalem]{ulem} % sout command
\usepackage{txfonts}
\DeclareMathAlphabet{\mathsc}{OT1}{cmr}{m}{sc}

\usepackage{listings}

\setbeamerfont{title}{size=\Huge, family=\bfseries\fosfamily}
\setbeamerfont{frametitle}{size=\LARGE, family=\bfseries\fosfamily}

\setbeamerfont{normal text}{size=\normalsize}
\setbeamerfont{itemize/enumerate body}{}
\setbeamerfont{itemize/enumerate subbody}{size=\small}
\setbeamerfont{itemize/enumerate subsubbody}{size=\footnotesize}

\usepackage{tudscrcolor}
\usepackage{environ}
\usepackage{tikz}
\usetikzlibrary{arrows,positioning,decorations.pathreplacing}
% Inspired by http://www.texample.net/tikz/examples/hand-drawn-lines/
\usetikzlibrary{decorations.pathmorphing}
\pgfdeclaredecoration{penciline}{initial}{
    \state{initial}[width=+\pgfdecoratedinputsegmentremainingdistance,
    auto corner on length=1mm,]{
        \pgfpathcurveto%
        {% From
            \pgfqpoint{\pgfdecoratedinputsegmentremainingdistance}
            {\pgfdecorationsegmentamplitude}
        }
        {%  Control 1
            \pgfmathrand
            \pgfpointadd{\pgfqpoint{\pgfdecoratedinputsegmentremainingdistance}{0pt}}
            {\pgfqpoint{-\pgfdecorationsegmentaspect
                    \pgfdecoratedinputsegmentremainingdistance}%
                {\pgfmathresult\pgfdecorationsegmentamplitude}
            }
        }
        {%TO
            \pgfpointadd{\pgfpointdecoratedinputsegmentlast}{\pgfpoint{1pt}{1pt}}
        }
    }
    \state{final}{}
}
\tikzset{handdrawn/.style={decorate,decoration=penciline}}
\tikzset{every shadow/.style={fill=none,shadow xshift=0pt,shadow yshift=0pt}}

\NewEnviron{doodlebox}[2]{%
    \begin{tikzpicture}[decoration=penciline, decorate]%
    \pgfmathsetseed{1237}%
    \node (n1) [decorate,draw=#1, fill=#2,thick,align=justify, text width=0.97\textwidth, inner ysep=2mm, inner xsep=2mm] at (0,0) {\BODY};%
    \end{tikzpicture}%
}
\NewEnviron{doodle}[1]{%
    \begin{tikzpicture}[decoration=penciline, decorate]%
    \pgfmathsetseed{1237}%
    \node (n1) [decorate,draw=#1, fill=#1!10,thick,align=justify, text width=0.97\textwidth, inner ysep=2mm, inner xsep=2mm] at (0,0) {\BODY};%
    \end{tikzpicture}%
}

\newcommand{\defineTitle}[3]{%
    \newcommand{\lectureindex}{#1}%
    \title{Programmierung}%
    \subtitle{Übung #1: #2}%
    \author{Eric Kunze \\ \url{eric.kunze@mailbox.tu-dresden.de} }%
    \date{#3}%
    \datecity{TU Dresden}%
}

\defineTitle{1}{Einführung in Haskell}{\today}
\renewcommand{\emph}[1]{\textbf{#1}}
\newcommand{\coloremph}[1]{\textcolor{cdpurple}{#1}}
\DeclareMathSymbol{*}{\mathbin}{symbols}{"01}
    
\begin{document}
    \maketitle
    
    \begin{frame}\frametitle{Allgemeine Hinweise}
        \begin{minipage}{\dimexpr0.5\linewidth-\fboxrule-\fboxsep}
            \emph{Vorlesung:}
            \begin{itemize}
                \item Freitag, 2. DS (9:20 - 10:50), HSZ/0003
                \item Skript und Aufgabensammlung im Copyshop \glqq Die Kopie\grqq
            \end{itemize}
            
            \medskip
            
            \emph{Übungen:}
            \begin{itemize}
                \item meine Übungsgruppen: \\
                Donnerstag, 1. DS und Freitag, 4. DS
            \end{itemize}
           \end{minipage}
           \pause
           \begin{minipage}{\dimexpr0.5\linewidth-\fboxrule-\fboxsep}
            \begin{doodle}{cddarkblue}
                \emph{Github:} \\
                \url{https://github.com/oakoneric/programmierung-ss19}
            \end{doodle}
        \end{minipage}
        \pause
        \begin{doodle}{cdorange}
            {\normalsize \bfseries Verlegungen:} \par 
            \smallskip
            Uns betreffen zwei Feiertage: \quad
            \emph{Karfreitag} (Freitag, 19.04.) und \emph{Himmelfahrt} (Donnerstag, 30.05.) \par \smallskip
            
            Mögliche Alternativtermine:
            \begin{itemize}
                \item Montag, 3. DS
                \item Montag, 5. DS
                \item Mittwoch, 2. DS
                \item Donnerstag, 2. DS (gerade Woche, insbesondere am 18.04.)
            \end{itemize}
        \end{doodle}  
    \end{frame}

        
%    \begin{frame}\frametitle{Haskell \& funktionale Programmierung}
%        Haskell ist im Gegensatz zu beispielsweise $C$ eine funktionale Programmiersprache. Wir programmieren also nicht \textit{wie} berechnet wird, sondern \textit{was} berechnet wird. \\
%        \medskip
%        \emph{Mathe:} Die Addition $n + m$ können wir auch als Funktion  
%        \begin{align*}
%            + \colon \mathbb{N} \times \mathbb{N} \to \mathbb{N} \qquad
%            n+m = 
%            \begin{cases}
%            n & m=0 \\
%            n + (m-1) &\text{sonst} \\
%            \end{cases}
%        \end{align*}
%        definieren. \\
%       
%        \medskip
%        
%        \emph{Informatik:} Aus \texttt{z = n + m} in $C$ können wir eine Funktion \texttt{add :: Int -> Int, add n 0 = n, add n m = 1 + add n (m-1)} spezifizieren, die genau die Addition durchführt.
%%        \begin{lstlisting}[language=Haskell]
%%            add :: Int -> Int
%%            add n 0 = n
%%            add n m = 1 + add n (m-1)
%%        \end{lstlisting}
%
%        \medskip
%        $\Longrightarrow$ Wir programmieren also \textit{Funktionen} und nähern uns daher der Mathematik an.
%    \end{frame}

    \begin{frame}\frametitle{Haskell installieren und compilieren}
        \begin{itemize}
            \item \emph{Glasgow Haskell Compiler} (ghc(i)) : \\ \url{https://www.haskell.org/ghc/}
            \item Ubuntu: \\
            \texttt{sudo apt install ghc}
            
            \medskip
            
            \item \emph{Terminal:} \\
            \texttt{ghci <modulname>}
            \item \emph{Module laden: } \\
            \texttt{:load <modulname>}
        \end{itemize}
    \end{frame}

    \begin{frame}\frametitle{Haskell \& Listen}
        \begin{itemize}
            \item Wenn \texttt{a} ein Typ ist, dann bezeichnet \texttt{[a]} den Typ ``Liste mit Elementen vom Typ \texttt{a}'', insbesondere
            haben alle Elemente einer Liste den gleichen Typ
            \bigskip \pause
            \item \emph{cons-Operator `` \texttt{:} ''} \qquad Trennung von head und tail einer Liste \\
            \texttt{[ x1 , x2 , x3 , x4 , x5] = x1 : [x2 , x3 , x4 , x5]}
            \bigskip \pause
            \item \emph{Verkettungsoperator `` \texttt{++} ''} \qquad Verkettung zweier Listen gleichen Typs \\
            \texttt{[x1 , x2 , x3] ++ [x4 , x5 , x6 , x7] = [x1 , x2 , x3 , x4 , x5 , x6 , x7]}
        \end{itemize}
    \end{frame}

    \begin{frame}\frametitle{Das Prinzip der Rekursion}
        \begin{doodle}{cdorange}
            \emph{Satz.}
            Sei $B$ eine Menge, $b \in B$ und $F \colon B \times \mathbb{N} \to B$ eine Abbildung. Dann liefert die Vorschrift
            \begin{subequations}
                \begin{align}
                    f(0) &:= b \\
                    f(n+1) &:= F(f(n),n) \quad \forall n \in \mathbb{N} \label{eq: rekursionsschritt}
                \end{align}
            \end{subequations}
            genau eine Abbildung $f : \mathbb{N} \to B$.
        \end{doodle} \\
        \bigskip
        \pause
        \emph{Beweis.} \qquad vollständige Induktion:
        \begin{description}
            \item[(IA)] Für $n=0$ ist $f(0) = b$ eindeutig definiert.
            \item[(IS)] Angenommen $f(n)$ sei eindeutig definiert. Wegen (\ref{eq: rekursionsschritt}) ist dann auch $f(n+1)$ eindeutig definiert.
        \end{description}
        
        \bigskip
        
        Man kann dieses Prinzip der Rekursion auf weitere Mengen (unabhängig von den natürlichen Zahlen) erweitern ($\nearrow$ Formale Systeme).
    \end{frame}

    \begin{frame}\frametitle{Aufgabe 1 -- Fakultätsfunktion}
        Fakultät
        \begin{equation}
            n! = \prod_{i=1}^n i
        \end{equation}
        
        \medskip
        \pause
        
        Daraus bilden wir nun eine Rekursionsvorschrift:
        \begin{equation}
           n! = \coloremph{\boldsymbol{n}} * \prod_{i=1}^{\coloremph{\boldsymbol{n-1}}} i= n * (n-1)!
        \end{equation}
        
        \medskip
        \pause
        
        Um die Rekursion vollständig zu definieren, benötigen wir einen (oder i.a. mehrere) \emph{Basisfall}. Wann können wir also die Rekursion der Fakultät abbrechen?
        \begin{equation}
            0 ! = 1 \qquad 1 ! = 1 \qquad 2! = 2 \qquad \cdots
        \end{equation}
        $\Rightarrow$ Welcher Basisfall ist sinnvoll? \qquad $0! = 1$.
    \end{frame}

    \begin{frame}\frametitle{Aufgabe 2 -- Fibonacci-Zahlen}
        \begin{equation}
            f_n := \begin{cases}
            \enskip 1 & \text{falls } n = 0 \\
            \enskip 1 & \text{falls } n = 1 \\
            \enskip f_{n-1} + f_{n-2} & \text{sonst}
            \end{cases}
        \end{equation}
        
        \medskip
        \pause
        
        $\Rightarrow$ Rekursionsvorschrift schon gegeben. \\
        
         \bigskip 
         \begin{minipage}{\dimexpr0.5\linewidth-\fboxrule-\fboxsep}
             \emph{Verfahren ohne Rekursion.} \\ \smallskip
             
             \begin{tikzpicture}[decoration=penciline, decorate]%
             \pgfmathsetseed{1237}%
             \node (n1) [decorate,draw=cdindigo, fill=cdindigo!10,thick,align=justify, , inner ysep=2mm, inner xsep=2mm, rectangle] at (0,0) {$f_{i-2}$};%
             \node (n2) [decorate,draw=cdindigo, fill=cdindigo!10,thick,align=justify, , inner ysep=2mm, inner xsep=2mm, rectangle] at (1.5,0) {$f_{i-1}$};%
             \node (n3) [decorate,draw=cdindigo, fill=cdindigo!10,thick,align=justify, , inner ysep=2mm, inner xsep=2mm, rectangle] at (3,0) {$n$};%
             \node (n4) [decorate,draw=cdindigo, fill=cdindigo!10,thick,align=justify, , inner ysep=2mm, inner xsep=2mm, rectangle] at (0,-1) {$f_{i-1}$};%
             \node (n5) [decorate,draw=cdindigo, fill=cdindigo!10,thick,align=justify, , inner ysep=2mm, inner xsep=2mm, rectangle] at (1.5,-1) {$f_{i-2} + f_{i-1}$};%
             \node (n6) [decorate,draw=cdindigo, fill=cdindigo!10,thick,align=justify, , inner ysep=2mm, inner xsep=2mm, rectangle] at (3,-1) {$n-1$};%
             \path[->] (n2) edge [bend left=0] (n4);
             \end{tikzpicture}%
         \end{minipage}
         \pause
         \begin{minipage}{\dimexpr0.5\linewidth-\fboxrule-\fboxsep}
             \emph{Anmerkung.} \\ \smallskip
             
             Explizite Formel:
             \begin{equation}
             f_n = \frac{\Phi^n - \left(-\frac{1}{\Phi} \right)^n}{\sqrt{5}} \qquad \text{ mit } \qquad \Phi = \frac{1 + \sqrt{5}}{2}
             \end{equation}
         \end{minipage}
    \end{frame}
\end{document}