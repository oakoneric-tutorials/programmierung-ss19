\documentclass[aspectratio=1610,onlymath, ngerman]{beamer}
% \documentclass[aspectratio=1610,onlymath,handout]{beamer}
\usepackage[ngerman]{babel}
\usepackage[utf8]{inputenc}
\usepackage[T1]{fontenc}

\usetheme[nosectionnum,pagenum,sansmath, cd2018, nodin]{tud}

%\usefonttheme[onlymath]{serif}
\usepackage{opensans}
\usepackage{stmaryrd}
\usepackage[normalem]{ulem} % sout command
\usepackage{txfonts}
\DeclareMathAlphabet{\mathsc}{OT1}{cmr}{m}{sc}

\usepackage{listings}
\lstset{language=Haskell,basicstyle=\ttfamily}
\usepackage{verbatim}
\usepackage{bold-extra}

\setbeamerfont{title}{size=\Huge, family=\bfseries\fosfamily}
\setbeamerfont{frametitle}{size=\LARGE, family=\bfseries\fosfamily}

\setbeamerfont{normal text}{size=\normalsize}
\setbeamerfont{itemize/enumerate body}{}
\setbeamerfont{itemize/enumerate subbody}{size=\small}
\setbeamerfont{itemize/enumerate subsubbody}{size=\footnotesize}

\usepackage{tudscrcolor}
\usepackage{environ}
\usepackage{tikz}
\usetikzlibrary{arrows,positioning,decorations.pathreplacing}
% Inspired by http://www.texample.net/tikz/examples/hand-drawn-lines/
\usetikzlibrary{decorations.pathmorphing}
\pgfdeclaredecoration{penciline}{initial}{
    \state{initial}[width=+\pgfdecoratedinputsegmentremainingdistance,
    auto corner on length=1mm,]{
        \pgfpathcurveto%
        {% From
            \pgfqpoint{\pgfdecoratedinputsegmentremainingdistance}
            {\pgfdecorationsegmentamplitude}
        }
        {%  Control 1
            \pgfmathrand
            \pgfpointadd{\pgfqpoint{\pgfdecoratedinputsegmentremainingdistance}{0pt}}
            {\pgfqpoint{-\pgfdecorationsegmentaspect
                    \pgfdecoratedinputsegmentremainingdistance}%
                {\pgfmathresult\pgfdecorationsegmentamplitude}
            }
        }
        {%TO
            \pgfpointadd{\pgfpointdecoratedinputsegmentlast}{\pgfpoint{1pt}{1pt}}
        }
    }
    \state{final}{}
}
\tikzset{handdrawn/.style={decorate,decoration=penciline}}
\tikzset{every shadow/.style={fill=none,shadow xshift=0pt,shadow yshift=0pt}}

\NewEnviron{doodlebox}[2]{%
    \begin{tikzpicture}[decoration=penciline, decorate]%
    \pgfmathsetseed{1237}%
    \node (n1) [decorate,draw=#1, fill=#2,thick,align=justify, text width=0.97\textwidth, inner ysep=2mm, inner xsep=2mm] at (0,0) {\BODY};%
    \end{tikzpicture}%
}
\NewEnviron{doodle}[1]{%
    \begin{tikzpicture}[decoration=penciline, decorate]%
    \pgfmathsetseed{1237}%
    \node (n1) [decorate,draw=#1, fill=#1!10,thick,align=justify, text width=0.97\textwidth, inner ysep=2mm, inner xsep=2mm] at (0,0) {\BODY};%
    \end{tikzpicture}%
}

\newcommand{\defineTitle}[3]{%
    \newcommand{\lectureindex}{#1}%
    \title{Programmierung}%
    \subtitle{Übung #1: #2}%
    \author{Eric Kunze \\ \url{eric.kunze@mailbox.tu-dresden.de} }%
    \date{#3}%
    \datecity{TU Dresden}%
}

\defineTitle{2}{Haskell -- Funktionen höherer Ordnung und Zeichenketten}{\today}
\renewcommand{\emph}[1]{\textbf{#1}}
\newcommand{\coloremph}[1]{\textcolor{cdpurple}{#1}}
\DeclareMathSymbol{*}{\mathbin}{symbols}{"01}
    
\begin{document}
    \maketitle
    
    \begin{frame}\frametitle{Rekapitulation: Funktionsprinzip}
    \normalsize
    	Wir erinnern uns an das \emph{Funktionsprinzip}: \\ \medskip
    	
    	Definition durch \\
    	\smallskip
    	\begin{itemize}
    		\item \emph{Basisfall.} $\quad$ $0$-äre Wertkonstruktoren \\
    		z.B. \texttt{[ \enskip ]} oder \texttt{Leaf} 
    		\medskip
    		\item \emph{Rekursionsfall.} $\quad$ $(>0)$-äre Wertkonstruktoren \\
    		z.B. \texttt{:} oder \texttt{Node}
    	\end{itemize}
    \end{frame}

    \begin{frame}\frametitle{Wichtige Hinweise}
    \normalsize
   		\emph{korrekte Klammerung}:
   		\begin{center}
   			\texttt{ f (Leaf a) (Leaf b)} $\quad$ statt $\quad$ \texttt{f Leaf a Leaf b}
   		\end{center}
   		
   		\bigskip
   		
	    \emph{Jeder Ausdruck hat einen Typ}: \\ \smallskip
   		Statt 
   		\begin{center}
   			\texttt{if a == b then boolExp else False}
   		\end{center}
   		sollte immer 
   		\begin{center}
   			\texttt{a == b \&\& boolExp}
   		\end{center}
		verwendet werden.
		\bigskip
		
%		\emph{Typ- und Wertkonstruktoren} sind unterschiedliche Dinge
    \end{frame}

	\begin{frame} \frametitle{Funktionen}
	\small
		Wir kennen bereits einige Möglichkeiten Funktionen zu notieren. Hier seien einige weitere erwähnt. \\
		\bigskip
		\emph{anonyme Funktionen.} \\
		Funktionen ohne konkreten Namen \\
		z.B. \texttt{(\textbackslash \!\!\! x -> x + 1)} ist die Addition mit \texttt{1} \\
		\medskip \pause
		\emph{Funktionskomposition.}
		Analog zur mathematischen Notation $f = g \circ h$ für $f(x) = g(h(x))$ versteht auch Haskell das Kompositionsprinzip mit dem Operator \texttt{.} \\ \smallskip
		z.B. 
		\begin{center}
			\texttt{prodOdd = prod . filter odd}
		\end{center} 
		statt \texttt{prodOdd xs = prod (filter odd xs)} für das Produkt aller ungeraden Listenelemente
		
	\end{frame}
    \begin{frame}\frametitle{Funktionen höherer Ordnung -- \texttt{map}}
    \normalsize 
    	{\Large Die Funktion \emph{\texttt{map}}.} \\ \medskip
    	\begin{itemize}
    		\item 
    		\begin{ttfamily}
    			map :: (a -> b) -> [a] -> [b] \\
	    		map f [] = [] \\
	    		map f (x:xs) = f x : map f xs
    		\end{ttfamily}
    		\pause \smallskip
    		\item \texttt{map} $\quad$ ermöglicht es eine Funktion \texttt{f} auf alle Elemente einer Liste anzuwenden
    		\pause \smallskip
    		\item \emph{Beispiel.} \\
    		 \texttt{map square [1,2,7,12,3,20] = [1,4,49,144,9,400]}
    	\end{itemize}
    \end{frame}

	\begin{frame}\frametitle{Funktionen höherer Ordnung -- \texttt{filter}}
	\normalsize 
	{\Large Die Funktion \emph{\texttt{filter}}.} \\ \medskip
	\begin{itemize}
		\item 
		\begin{ttfamily}
			filter :: (a -> Bool) -> [a] -> [a] \\
			filter p xs = [ x | x <- xs, p x]
		\end{ttfamily}
		\pause \smallskip
		\item \texttt{filter p xs} $\quad$ liefert eine Liste, die genau die Elemente von \texttt{xs} enthält, welche das Prädikat \texttt{p} erfüllen
		\pause \smallskip
		\item \emph{Beispiel.} \\
		\texttt{filter odd [1,2,7,12,3,20] = [1,7,3]}
	\end{itemize}
	\end{frame}

	\begin{frame}\frametitle{Funktionen höherer Ordnung -- \texttt{foldr}}
		\normalsize 
		{\Large Die Funktion \emph{\texttt{foldr}}.} \\ \medskip
		\begin{itemize}
			\item 
			\begin{ttfamily}
				foldr :: (a -> b -> b) -> b -> [a] -> b \\
				foldr f z []     = z \\
				foldr f z (x:xs) = f x (foldr f z xs) 
			\end{ttfamily}
			\pause \smallskip
			\item \texttt{foldr f z xs} $\quad$ faltet eine Liste \texttt{xs} und verknüpft jeweils durch die Funktion \texttt{f}; gestartet wird mit \texttt{z} und dem rechtesten Element
			\pause \smallskip
			\item \emph{Beispiel.} \\
			\texttt{foldr (+) 3 [1,2,3,4,5] = 18} \\
			\texttt{length xs = foldr (+) 0 (map (\textbackslash \!\!\! x -> 1) xs) }
			\pause \smallskip
			\item Analog existiert auch eine Funktion \texttt{foldl}, die eine Liste von links beginnend faltet.
		\end{itemize}
	\end{frame}


	\begin{frame}\frametitle{Funktionen höherer Ordnung -- Übersicht}
		\normalsize 
		\begin{itemize}
			\item \texttt{\textbf{map}} wendet Funktion auf alle Listenelemente an \\
				\begin{ttfamily}
					map :: (a -> b) -> [a] -> [b] \\
					map f [] = [] \\
					map f (x:xs) = f x : map f xs
				\end{ttfamily}
			\medskip
			\item \texttt{\textbf{filter}} wählt Listenelemente anhand einer Funktion aus \\
				\begin{ttfamily}
					filter :: (a -> Bool) -> [a] -> [a] \\
					filter p xs = [ x | x <- xs, p x]
				\end{ttfamily}
			\medskip
			\item \texttt{\textbf{foldr}} faltet eine Liste mit Verknüpfungsfunktion (von rechts beginnend) \\
				\begin{ttfamily}
					foldr :: (a -> b -> b) -> b -> [a] -> b \\
					foldr f z []     = z \\
					foldr f z (x:xs) = f x (foldr f z xs) 
				\end{ttfamily}
		\end{itemize}
	\end{frame}

	\begin{frame} \frametitle{Zeichenketten}
	\normalsize
		\emph{Characters} werden in \texttt{' '} eingeschlossen. \\
		\emph{Strings} werden in \texttt{`` ''} eingeschlossen. $\quad \rightarrow$ normale double-quotes \\
		\bigskip \pause
		Dabei gilt \texttt{String = [Char]}. \\
		\bigskip \pause
		Überprüfe den Typ in GHCi anhand eigener Beispiele mittels \texttt{:t}.\\
		z.B. \\
		\lstinline{:t ['a','b']} $\Longrightarrow$ \lstinline{['a','b'] :: [Char]} \\
		\lstinline{:t 'a'}      $\Longrightarrow$ \lstinline{'a' :: Char} \\
		\lstinline{:t ``ab''}     $\Longrightarrow$ \lstinline{``ab'' :: [Char]} 
	\end{frame}


\end{document}