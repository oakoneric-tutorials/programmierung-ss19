\usepackage[ngerman]{babel}
\usepackage[utf8]{inputenc}
\usepackage[T1]{fontenc}

\usetheme[nosectionnum,pagenum,sansmath, cd2018, nodin]{tud}

%\usefonttheme[onlymath]{serif}
\usepackage{opensans}
\usepackage{stmaryrd}
\usepackage[normalem]{ulem} % sout command
\usepackage{txfonts}
\DeclareMathAlphabet{\mathsc}{OT1}{cmr}{m}{sc}

\usepackage{listings}
\lstset{language=Haskell,basicstyle=\ttfamily}
\usepackage{verbatim}
\usepackage{bold-extra}

\setbeamerfont{title}{size=\Huge, family=\bfseries\fosfamily}
\setbeamerfont{frametitle}{size=\LARGE, family=\bfseries\fosfamily}

\setbeamerfont{normal text}{size=\normalsize}
\setbeamerfont{itemize/enumerate body}{}
\setbeamerfont{itemize/enumerate subbody}{size=\small}
\setbeamerfont{itemize/enumerate subsubbody}{size=\footnotesize}

\usepackage{tudscrcolor}
\usepackage{environ}
\usepackage{tikz}
\usetikzlibrary{arrows,positioning,decorations.pathreplacing}
% Inspired by http://www.texample.net/tikz/examples/hand-drawn-lines/
\usetikzlibrary{decorations.pathmorphing}
\pgfdeclaredecoration{penciline}{initial}{
    \state{initial}[width=+\pgfdecoratedinputsegmentremainingdistance,
    auto corner on length=1mm,]{
        \pgfpathcurveto%
        {% From
            \pgfqpoint{\pgfdecoratedinputsegmentremainingdistance}
            {\pgfdecorationsegmentamplitude}
        }
        {%  Control 1
            \pgfmathrand
            \pgfpointadd{\pgfqpoint{\pgfdecoratedinputsegmentremainingdistance}{0pt}}
            {\pgfqpoint{-\pgfdecorationsegmentaspect
                    \pgfdecoratedinputsegmentremainingdistance}%
                {\pgfmathresult\pgfdecorationsegmentamplitude}
            }
        }
        {%TO
            \pgfpointadd{\pgfpointdecoratedinputsegmentlast}{\pgfpoint{1pt}{1pt}}
        }
    }
    \state{final}{}
}
\tikzset{handdrawn/.style={decorate,decoration=penciline}}
\tikzset{every shadow/.style={fill=none,shadow xshift=0pt,shadow yshift=0pt}}

\NewEnviron{doodlebox}[2]{%
    \begin{tikzpicture}[decoration=penciline, decorate]%
    \pgfmathsetseed{1237}%
    \node (n1) [decorate,draw=#1, fill=#2,thick,align=justify, text width=0.97\textwidth, inner ysep=2mm, inner xsep=2mm] at (0,0) {\BODY};%
    \end{tikzpicture}%
}
\NewEnviron{doodle}[1]{%
    \begin{tikzpicture}[decoration=penciline, decorate]%
    \pgfmathsetseed{1237}%
    \node (n1) [decorate,draw=#1, fill=#1!10,thick,align=justify, text width=0.97\textwidth, inner ysep=2mm, inner xsep=2mm] at (0,0) {\BODY};%
    \end{tikzpicture}%
}

\newcommand{\defineTitle}[3]{%
    \newcommand{\lectureindex}{#1}%
    \title{Programmierung}%
    \subtitle{Übung #1: #2}%
    \author{Eric Kunze \\ \url{eric.kunze@mailbox.tu-dresden.de} }%
    \date{#3}%
    \datecity{TU Dresden}%
}