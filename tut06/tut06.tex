\documentclass[aspectratio=1610,onlymath, ngerman]{beamer}
% \documentclass[aspectratio=1610,onlymath,handout]{beamer}
\usepackage[ngerman]{babel}
\usepackage[utf8]{inputenc}
\usepackage[T1]{fontenc}

\usetheme[nosectionnum,pagenum,sansmath, cd2018, nodin]{tud}

%\usefonttheme[onlymath]{serif}
\usepackage{opensans}
\usepackage{stmaryrd}
\usepackage[normalem]{ulem} % sout command
\usepackage{txfonts}
\DeclareMathAlphabet{\mathsc}{OT1}{cmr}{m}{sc}

\usepackage{listings}
\lstset{language=Haskell,basicstyle=\ttfamily}
\usepackage{verbatim}
\usepackage{bold-extra}

\setbeamerfont{title}{size=\Huge, family=\bfseries\fosfamily}
\setbeamerfont{frametitle}{size=\LARGE, family=\bfseries\fosfamily}

\setbeamerfont{normal text}{size=\normalsize}
\setbeamerfont{itemize/enumerate body}{}
\setbeamerfont{itemize/enumerate subbody}{size=\small}
\setbeamerfont{itemize/enumerate subsubbody}{size=\footnotesize}

\usepackage{tudscrcolor}
\usepackage{environ}
\usepackage{tikz}
\usetikzlibrary{arrows,positioning,decorations.pathreplacing}
% Inspired by http://www.texample.net/tikz/examples/hand-drawn-lines/
\usetikzlibrary{decorations.pathmorphing}
\pgfdeclaredecoration{penciline}{initial}{
    \state{initial}[width=+\pgfdecoratedinputsegmentremainingdistance,
    auto corner on length=1mm,]{
        \pgfpathcurveto%
        {% From
            \pgfqpoint{\pgfdecoratedinputsegmentremainingdistance}
            {\pgfdecorationsegmentamplitude}
        }
        {%  Control 1
            \pgfmathrand
            \pgfpointadd{\pgfqpoint{\pgfdecoratedinputsegmentremainingdistance}{0pt}}
            {\pgfqpoint{-\pgfdecorationsegmentaspect
                    \pgfdecoratedinputsegmentremainingdistance}%
                {\pgfmathresult\pgfdecorationsegmentamplitude}
            }
        }
        {%TO
            \pgfpointadd{\pgfpointdecoratedinputsegmentlast}{\pgfpoint{1pt}{1pt}}
        }
    }
    \state{final}{}
}
\tikzset{handdrawn/.style={decorate,decoration=penciline}}
\tikzset{every shadow/.style={fill=none,shadow xshift=0pt,shadow yshift=0pt}}

\NewEnviron{doodlebox}[2]{%
    \begin{tikzpicture}[decoration=penciline, decorate]%
    \pgfmathsetseed{1237}%
    \node (n1) [decorate,draw=#1, fill=#2,thick,align=justify, text width=0.97\textwidth, inner ysep=2mm, inner xsep=2mm] at (0,0) {\BODY};%
    \end{tikzpicture}%
}
\NewEnviron{doodle}[1]{%
    \begin{tikzpicture}[decoration=penciline, decorate]%
    \pgfmathsetseed{1237}%
    \node (n1) [decorate,draw=#1, fill=#1!10,thick,align=justify, text width=0.97\textwidth, inner ysep=2mm, inner xsep=2mm] at (0,0) {\BODY};%
    \end{tikzpicture}%
}

\newcommand{\defineTitle}[3]{%
    \newcommand{\lectureindex}{#1}%
    \title{Programmierung}%
    \subtitle{Übung #1: #2}%
    \author{Eric Kunze \\ \url{eric.kunze@mailbox.tu-dresden.de} }%
    \date{#3}%
    \datecity{TU Dresden}%
}

%%%%%%%%%%%%%%%%%%%%%%%%%%%%%%%%%%%%%%%%%%%%%%%%%%%%%%%%%%%%%%%%%%%%%%%%%%%%
%%%%%%%%%%%%%%%%%%%%%%%%%%%%%%%%%%%%%%%%%%%%%%%%%%%%%%%%%%%%%%%%%%%%%%%%%%%%

\defineTitle{5}{Unifikation \& Induktion auf Listen}{\today}

\renewcommand{\emph}[1]{\textbf{#1}}
\newcommand{\coloremph}[1]{\textcolor{cdpurple}{#1}}
\newcommand{\col}[1]{\textcolor{cdpurple}{\boldsymbol{#1}}}

\DeclareMathSymbol{*}{\mathbin}{symbols}{"01}

\renewcommand*{\headerinfo}{\color{cdgray}\textbf{Github:} \url{https://github.com/oakoneric/programmierung-ss19}}
\arraycolsep2pt

\usepackage{aligned-overset}
\newcommand{\cw}[1]{\texttt{#1}}
\newcommand{\step}[2][]{\ensuremath{\overset{{#1} (\text{#2})}{=}}}
\newcommand*{\astep}[2][]{\ensuremath{\overset{{#1} (\text{#2})}&{=}}}

\begin{document}
    \maketitle


	\begin{frame}\frametitle{Unifikationsalgorithmus -- Regeln}
	\small
		\begin{itemize}
			\item \emph{Dekomposition.} Sei $\delta \in \Sigma$ ein $k$-stelliger Konstruktor, $s_1 , \dots , s_k , t_1 , \dots , t_k$ Terme über Konstruktoren und Variablen.
			\begin{equation*}
				\begin{pmatrix}
				\delta(s_1, \dots , s_k) \\ \delta(t_1, \dots , t_k)
				\end{pmatrix}
				\quad \leadsto \quad
				\begin{pmatrix} s_1 \\ t_1 \end{pmatrix} , \dots , \begin{pmatrix} s_k \\ t_k \end{pmatrix}
			\end{equation*}
			\item \emph{Elimination.} Sei $x$ eine Variable \coloremph{!}
			\begin{equation*}
				\begin{pmatrix} x \\ x \end{pmatrix}
				\quad \leadsto \quad
				\emptyset
			\end{equation*}
			\item \emph{Vertauschung.} Sei $t$ keine Variable.
			\begin{equation*}
				\begin{pmatrix} t \\ x \end{pmatrix}
				\quad \leadsto \quad 
				\begin{pmatrix} x \\ t \end{pmatrix}
			\end{equation*}
			\item \emph{Substitution.} Sei $x$ eine Variable, $t$ keine Variable und $x$ kommt nicht in $t$ vor (occur check). Dann ersetze in jedem anderen Term die Variable $x$ durch $t$.
		\end{itemize}
	\end{frame}

	\begin{frame}\frametitle{Aufgabe 1}
		\begin{minipage}{\dimexpr0.5\linewidth-\fboxrule-\fboxsep}
		\begin{align*}
			&\left\{\left(\begin{array}{lcrlcl}
				\col{\delta(} \alpha \col{,} & \sigma(x_1 , \alpha) \col{,} & \sigma(x_2,x_3) \col{)} \\
				\col{\delta(} \alpha \col{,} & \sigma(x_1, x_2)     \col{,} & \sigma(x_2,\gamma(x_2)\col{)}
				\end{array}\right)\right\}\\
			%
			\overset{\text{Dek.}}{\Longrightarrow}
			%
			&\left\{ \begin{pmatrix}
			\col{\alpha} \\ \col{\alpha}
			\end{pmatrix}, \begin{pmatrix}
			\col{\sigma(} x_1 \col{,} \alpha \col{)} \\ \col{\sigma}(x_1 \col{,} x_2 \col{)}
			\end{pmatrix}, \begin{pmatrix}
			\col{\sigma(} x_2 \col{,} x_3 \col{)} \\ \col{\sigma(} x_2 \col{,} \gamma(x_2) \col{)}
			\end{pmatrix}
			\right\}\\
			%
			\overset{\text{Dek.}}{\Longrightarrow^3}
			%
			&\left\{ \begin{pmatrix}
			\col{x_1} \\ \col{x_1}
			\end{pmatrix}, \begin{pmatrix}
			\alpha \\ x_2
			\end{pmatrix}, \begin{pmatrix}
			\col{x_2} \\ \col{x_2}
			\end{pmatrix}, \begin{pmatrix}
			x_3 \\ \gamma(x_2)
			\end{pmatrix}
			\right\}\\
			%
			\overset{\text{El.}}{\Longrightarrow^2}
			%
			&\left\{ \begin{pmatrix}
			\col{\alpha} \\ \col{x_2}
			\end{pmatrix}, \begin{pmatrix}
			x_3 \\ \gamma(x_2)
			\end{pmatrix}
			\right\}\\
			%
			\overset{\text{Vert.}}{\Longrightarrow}
			%
			&\left\{ \begin{pmatrix}
			\col{x_2} \\ \alpha
			\end{pmatrix}, \begin{pmatrix}
			x_3 \\ \gamma(\col{x_2})
			\end{pmatrix}
			\right\}\\	
			%
			\overset{\text{Subst.}}{\Longrightarrow}
			%
			&\left\{ \begin{pmatrix}
			x_2 \\ \alpha
			\end{pmatrix}, \begin{pmatrix}
			x_3 \\ \gamma(\alpha)
			\end{pmatrix}
			\right\}\\	
		\end{align*}
		\end{minipage}
		\pause
		\begin{minipage}{\dimexpr0.5\linewidth-\fboxrule-\fboxsep}
		\begin{center}
		\emph{allgemeinster Unifikator:} \vspace{-\baselineskip}
			\begin{align*}
				x_1 \mapsto x_1 \qquad
				x_2 \mapsto \alpha \qquad
				x_3 \mapsto \gamma(\alpha)
			\end{align*}
%			
			\medskip
			\pause
						
			\emph{Teilaufgabe (b)} \vspace{-\baselineskip}
			\begin{align*}
				t_1 &= \texttt{(a , [a])} \\
				t_2 &= \texttt{(Int , [Double])} \\
				t_3 &= \texttt{(b , c)}
			\end{align*}
			\pause
			\begin{itemize}
				\item $t_1$ und $t_2$ sind \pause \emph{nicht} unifizierbar
				
				\item $t_1$ und $t_3$ sind \pause unifizierbar mit \pause
				$\texttt{a} \mapsto \texttt{a} \qquad \texttt{b} \mapsto \texttt{a} \qquad \texttt{c} \mapsto \texttt{[a]}$
				
				\item $t_2$ und $t_3$ sind \pause unifizierbar mit \pause 		$\texttt{b} \mapsto \texttt{Int} \qquad \texttt{c} \mapsto \texttt{[Double]}$
			\end{itemize}
		\end{center}
		\end{minipage}
	\end{frame}

	\begin{frame} \frametitle{Aufgabe 2}
	\begin{minipage}{\dimexpr0.5\linewidth-\fboxrule-\fboxsep}
		\begin{align*}
		&\left\{\left(\begin{array}{rclllcl}
		\col{\sigma} \, \col{\Big(} & \gamma(x_2) &\col{,} & \sigma ( & \gamma(\alpha) , & x_3                &) \col{\Big)} \\
		\col{\sigma} \, \col{\Big(} & x_1         &\col{,} & \sigma ( & \gamma(\alpha) , & \sigma(\alpha,x_1) &) \col{\Big)}
		\end{array}\right)\right\}\\
		%
		\overset{\text{Dek.}}{\Longrightarrow}
		%
		&\left\{\begin{pmatrix}
		\gamma(x_2) \\ x_1
		\end{pmatrix} , 
		\left(\begin{array}{llcl}
		\col{\sigma (} & \gamma(\alpha) \col{,} & x_3                &\col{)}  \\
		\col{\sigma (} & \gamma(\alpha) \col{,} & \sigma(\alpha,x_1) &\col{)}
		\end{array}\right)	\right\}\\
		%
		\overset{\text{Dek.}}{\Longrightarrow}
		%
		&\left\{\begin{pmatrix}
		\gamma(x_2) \\ x_1
		\end{pmatrix} , \begin{pmatrix}
		\col{\gamma(\alpha)} \\ \col{\gamma(\alpha)}
		\end{pmatrix} , \begin{pmatrix}
		x_3 \\ \sigma(\alpha,x_1)
		\end{pmatrix} \right\}\\
		%
		\overset{\text{Dek.}}{\Longrightarrow^2}
		%
		&\left\{ \begin{pmatrix}
		\col{\gamma(x_2)} \\ \col{x_1}
		\end{pmatrix} , \begin{pmatrix}
		x_3 \\ \sigma(\alpha,x_1)
		\end{pmatrix} \right\} \\
		%
		\overset{\text{Vert.}}{\Longrightarrow}
		%
		&\left\{ \begin{pmatrix}
		\col{x_1} \\ \gamma(x_2)
		\end{pmatrix} , \begin{pmatrix}
		x_3 \\ \sigma(\alpha,\col{x_1})
		\end{pmatrix} \right\} \quad {\tiny \textcolor{cdgray}{x_1 \text{ kommt nicht in } \gamma(x_2) \text{ vor}}}\\
		%
		\overset{\text{Sub.}}{\Longrightarrow}
		%
		&\left\{ \begin{pmatrix}
		x_1 \\ \gamma(x_2)
		\end{pmatrix} , \begin{pmatrix}
		x_3 \\ \sigma(\alpha,\gamma(x_2))
		\end{pmatrix} \right\}
		\end{align*}
	\end{minipage}
	\pause
	\begin{minipage}{\dimexpr0.5\linewidth-\fboxrule-\fboxsep}
		\begin{center}
			\emph{allgemeinster Unifikator:} \vspace{-\baselineskip}
			\begin{align*}
			x_1 \mapsto \gamma(x_2) \qquad
			x_2 \mapsto x_2 \qquad
			x_3 \mapsto \sigma(\alpha,\gamma(x_2))
			\end{align*}
			
			\medskip
			\pause
			
			\emph{weitere Unifikatoren:} \vspace{-\baselineskip}
			\begin{align*}
			x_1 &\mapsto \gamma(\alpha)
			&x_2 &\mapsto \alpha
			&x_3 &\mapsto \sigma(\alpha, \gamma(\alpha)) \\
			%
			x_1 &\mapsto \gamma(\gamma(\alpha))
			&x_2 &\mapsto \gamma(\alpha)
			&x_3 &\mapsto \sigma(\alpha , \gamma(\gamma(\alpha)))
			\end{align*}
		\end{center}
	\end{minipage}
\end{frame}

\begin{frame} \frametitle{Strukturelle Induktion}
\small
\begin{itemize}
	\item \emph{Allgemeine Hinweise.} Es müssen \coloremph{alle} Variablen quantifiziert werden!
	\medskip
	\item \emph{Was wollen wir?} Wir wollen \coloremph{zeigen}, dass eine Eigenschaft (ein Prädikat) $P$ für jede Liste \texttt{xs :: [a]} gilt, d.h. dass $P(\texttt{xs})$ gilt.
	\pause
	\bigskip
	\item \emph{Induktionsanfang.} Wir zeigen $P(\texttt{xs})$ für $\texttt{xs == [\,]}$. \\
	Achtung: hat P weitere freie Parameter, dann müssen auch diese quantifiziert werden!
	\pause
	\medskip	
	\item \emph{Induktionsvoraussetzung.} Wir nehmen an, dass $P(\texttt{xs'})$ für eine Liste $\texttt{xs' :: [a]}$ gilt. \\
	Achtung: freie Parameter!
	\pause
	\medskip
	\item \emph{Induktionsschritt.} Nutze die Induktionsvoraussetzung um zu zeigen, dass $P(\texttt{x : xs'})$ für alle $\texttt{x :: a}$ gilt.
\end{itemize}
\end{frame}

\begin{frame} \frametitle{Aufgabe 3}
\small
	\begin{itemize}
		\item \emph{zu zeigen.} 
		\begin{equation*}
			\cw{sum (foo xs) = 2 * sum xs - length xs} \qquad \text{für alle } \cw{xs :: Int}
		\end{equation*}
		
		\pause
		\medskip
		
		\item \emph{Induktionsanfang.} \quad Sei \cw{xs :: [Int]} mit \cw{xs == []}. 
		\begin{align*}
			\text{linke Seite: } \quad &\cw{sum (foo []) \step{2} sum [] \step{6} 0} \\
			\text{rechte Seite: } \quad &\cw{2 * sum [] - length [] \step{10} 2 * sum [] - 0 \step{6} 2 * 0 - 0 = 0}
		\end{align*}
		
		\pause
		\medskip
		
		\item \emph{Induktionsvoraussetzung.} \quad Sei \cw{xs :: [Int]}, sodass gilt
		\begin{equation*}
		\cw{sum (foo xs) = 2 * sum xs - length xs}
		\end{equation*}
	\end{itemize}
\end{frame}

\begin{frame} \frametitle{Aufgabe 3}
	\begin{itemize}
		\item \emph{Induktionsschritt.} \quad Für alle \cw{x :: Int} zeigen wir, dass gilt
		\begin{equation*}
		\cw{sum ( foo (x:xs) ) = 2 * sum (x:xs) - length (x:xs)}
		\end{equation*}
		\underline{Beweis.} \vspace{-0.75\baselineskip}
		\begin{align*}
			\cw{sum (foo (x:xs))} \astep{3} \cw{sum (x : x : (-1) : foo xs)} \\
			\astep[3*]{7} \cw{x + x + (-1) + sum (foo xs)} \\
			\astep{IV} \cw{x + x + (-1) + 2 * sum xs - length xs} \\
			\astep{Komm.} \cw{2 * x + 2 * sum xs - 1 - length xs)} \\
			\astep{Dist.} \cw{2 * (x + sum xs) - (1 + length xs)} \\
			\astep{7} \cw{2 * sum (x:xs) - (1 + length xs)} \\
			\astep{11} \cw{2 * sum (x:xs) - length (x:xs)} 
		\end{align*}
		\vspace{-3\baselineskip}\flushright\qed
	\end{itemize}
\end{frame}

\end{document}