\documentclass[aspectratio=1610,onlymath, ngerman]{beamer}
% \documentclass[aspectratio=1610,onlymath,handout]{beamer}
\usepackage[ngerman]{babel}
\usepackage[utf8]{inputenc}
\usepackage[T1]{fontenc}

\usetheme[nosectionnum,pagenum,sansmath, cd2018, nodin]{tud}

%\usefonttheme[onlymath]{serif}
\usepackage{opensans}
\usepackage{stmaryrd}
\usepackage[normalem]{ulem} % sout command
\usepackage{txfonts}
\DeclareMathAlphabet{\mathsc}{OT1}{cmr}{m}{sc}

\usepackage{listings}
\lstset{language=Haskell,basicstyle=\ttfamily}
\usepackage{verbatim}
\usepackage{bold-extra}

\setbeamerfont{title}{size=\Huge, family=\bfseries\fosfamily}
\setbeamerfont{frametitle}{size=\LARGE, family=\bfseries\fosfamily}

\setbeamerfont{normal text}{size=\normalsize}
\setbeamerfont{itemize/enumerate body}{}
\setbeamerfont{itemize/enumerate subbody}{size=\small}
\setbeamerfont{itemize/enumerate subsubbody}{size=\footnotesize}

\usepackage{tudscrcolor}
\usepackage{environ}
\usepackage{tikz}
\usetikzlibrary{arrows,positioning,decorations.pathreplacing}
% Inspired by http://www.texample.net/tikz/examples/hand-drawn-lines/
\usetikzlibrary{decorations.pathmorphing}
\pgfdeclaredecoration{penciline}{initial}{
    \state{initial}[width=+\pgfdecoratedinputsegmentremainingdistance,
    auto corner on length=1mm,]{
        \pgfpathcurveto%
        {% From
            \pgfqpoint{\pgfdecoratedinputsegmentremainingdistance}
            {\pgfdecorationsegmentamplitude}
        }
        {%  Control 1
            \pgfmathrand
            \pgfpointadd{\pgfqpoint{\pgfdecoratedinputsegmentremainingdistance}{0pt}}
            {\pgfqpoint{-\pgfdecorationsegmentaspect
                    \pgfdecoratedinputsegmentremainingdistance}%
                {\pgfmathresult\pgfdecorationsegmentamplitude}
            }
        }
        {%TO
            \pgfpointadd{\pgfpointdecoratedinputsegmentlast}{\pgfpoint{1pt}{1pt}}
        }
    }
    \state{final}{}
}
\tikzset{handdrawn/.style={decorate,decoration=penciline}}
\tikzset{every shadow/.style={fill=none,shadow xshift=0pt,shadow yshift=0pt}}

\NewEnviron{doodlebox}[2]{%
    \begin{tikzpicture}[decoration=penciline, decorate]%
    \pgfmathsetseed{1237}%
    \node (n1) [decorate,draw=#1, fill=#2,thick,align=justify, text width=0.97\textwidth, inner ysep=2mm, inner xsep=2mm] at (0,0) {\BODY};%
    \end{tikzpicture}%
}
\NewEnviron{doodle}[1]{%
    \begin{tikzpicture}[decoration=penciline, decorate]%
    \pgfmathsetseed{1237}%
    \node (n1) [decorate,draw=#1, fill=#1!10,thick,align=justify, text width=0.97\textwidth, inner ysep=2mm, inner xsep=2mm] at (0,0) {\BODY};%
    \end{tikzpicture}%
}

\newcommand{\defineTitle}[3]{%
    \newcommand{\lectureindex}{#1}%
    \title{Programmierung}%
    \subtitle{Übung #1: #2}%
    \author{Eric Kunze \\ \url{eric.kunze@mailbox.tu-dresden.de} }%
    \date{#3}%
    \datecity{TU Dresden}%
}

%%%%%%%%%%%%%%%%%%%%%%%%%%%%%%%%%%%%%%%%%%%%%%%%%%%%%%%%%%%%%%%%%%%%%%%%%%%%
%%%%%%%%%%%%%%%%%%%%%%%%%%%%%%%%%%%%%%%%%%%%%%%%%%%%%%%%%%%%%%%%%%%%%%%%%%%%

\defineTitle{7}{$\lambda$-Kalkül}{24.~Mai~2019}

\renewcommand{\emph}[1]{\textbf{#1}}
\newcommand{\coloremph}[1]{\textcolor{cdpurple}{#1}}
\newcommand{\col}[1]{\textcolor{cdpurple}{\boldsymbol{#1}}}
\newcommand{\coll}[1]{\textcolor{cddarkgreen}{\boldsymbol{#1}}}
\newcommand{\colll}[1]{\textcolor{cdorange}{\boldsymbol{#1}}}

\DeclareMathSymbol{*}{\mathbin}{symbols}{"01}

\renewcommand*{\headerinfo}{\color{cdgray}\textbf{Github:} \url{https://github.com/oakoneric/programmierung-ss19}}
\arraycolsep2pt

\usepackage{aligned-overset}
\newcommand{\cw}[1]{\texttt{#1}}
\newcommand{\step}[2][]{\ensuremath{\overset{{#1} (\text{#2})}{=}}}
\newcommand*{\astep}[2][]{\ensuremath{\overset{{#1} (\text{#2})}&{=}}}

\newcommand{\num}[1]{\ensuremath{\langle #1 \rangle}}

\begin{document}
    \maketitle


	\begin{frame}\frametitle{Der $\lambda$-Kalkül}
	\small
		\begin{itemize}
			\item \emph{Atome.} $\qquad \quad \enskip$ $x,y$
			\item \emph{Abstraktion.} $\quad$ $\lambda x.t$ \hspace{1cm} \textcolor{cdgray}{($f(x) = t$, anonyme Funktion)}
			\item \emph{Applikation.} $\quad \ $ $t_1 \enskip  t_2$
			\medskip
			\item \emph{Bsp.} $\num{\texttt{quadriere}} = \lambda x . \ast x x$ $\quad \hookrightarrow \quad$
			$\num{\texttt{quadriere}} \ \langle 2 \rangle = 2 * 2 = 4$
		\end{itemize}
	
	\bigskip
	
	\emph{Verabredungen:}
		\begin{itemize}
			\item Applikation ist linksassoziativ: $((t_1 \, t_2) \, t_3) = t_1 \, t_2 \, t_3 \enskip \text{für alle } t_1, t_2, t_3 \in \lambda(\Sigma)$
			\item mehrfache Abstraktion: $(\lambda x_1 . (\lambda x_2 . (\lambda x_3 . t)))) = (\lambda x_1 x_2 x_3 . t) \enskip \text{für alle } t \in \lambda(\Sigma)$
			\item Applikation vor Abstraktion: $(\lambda x. xy) = (\lambda x . (xy)) \neq ((\lambda x.x) y)$
		\end{itemize}
	\end{frame}

	\begin{frame} \frametitle{Aufgabe 1}
	\small
		\begin{tabbing}
		\emph{(a)} $\quad$ \= $A$ mit $A \ t \ s \ u \Rightarrow^\ast s$:
		\pause $\qquad$ \= $A = (\lambda xyz \ . \ y)$ \\[6pt]
		\pause
		\emph{(b)} \> $B$ mit $B \ t \ s \Rightarrow^\ast s \ t$: 
		\pause \> $B = (\lambda xy \ . \ yx)$ \\[6pt]
		\pause
		\emph{(c)} \> $C$ mit $C \ C \Rightarrow^\ast C \ C$: 
		\pause \> $C =( \lambda x \ . \ xx)$ 
		\pause $\enskip , \qquad$ denn: 
		$(\lambda x . \underbrace{xx}_{GV=\emptyset}) \underbrace{(\lambda x.xx)}_{FV = \emptyset} \quad \Rightarrow^{\beta} \quad (\lambda x . xx) (\lambda x.xx)$ \\[6pt]
		\pause
		\emph{(d)} \> $D$ mit $D \Rightarrow^\ast D$: 
		\pause \> $D = (C \ C)$ \\[6pt]
		\pause	
		\emph{(e)} \> $E$ mit $E \ E \ t \Rightarrow^\ast E \ t \ E$: 
		\pause \> $E = (\lambda xy \ . \ xyx)$ \pause $\enskip , \qquad$ denn:
		\end{tabbing}
		\begin{equation*}
			(\lambda x\underbrace{y \ . \ xyx}_{GV = \{y\}}) (\underbrace{\lambda xy \ . \ xyx}_{FV = \emptyset}) t 
			\pause \enskip \Rightarrow^{\beta} \enskip
			(\lambda y \ . \ (\lambda xy \ . \ xyx) \ y \ (\lambda xy \ . \ xyx)) \ t 
			\pause \enskip \Rightarrow^{\beta} \enskip 
			(\underbrace{\lambda xy \ . \ xyx}_{ = E}) \ t \ (\underbrace{\lambda xy \ . \ xyx}_{=E})
		\end{equation*}
	\end{frame}

	\begin{frame} \frametitle{Church-Numerals}
		\small
		Dadurch, dass wir im Folgenden keine Symbole mehr zulassen (d.h. $\Sigma = \emptyset$), benötigen wir eine alternative Charakterisierung dieser. \\
		Zuerst beschäftigen uns die natürlichen Zahlen. \\
		
		\medskip \pause
		 
		\emph{Darstellung der natürlichen Zahlen} $\rightarrow$ \coloremph{Church-Numerals} \\
		\begin{align*}
			\num{0} &= (\lambda xy \ . \ y) \\ 
			\num{1} &= (\lambda xy \ . \ xy) \\
			\num{2} &= (\lambda xy \ . \ x(xy)) \\
			&\vdots \\ \pause
			\num{n} &= (\lambda xy \ . \ \underbrace{x (x \dots (x}_n y ) \dots ))
		\end{align*}
		
	\end{frame}
	\begin{frame} \frametitle{Aufgabe 2}
	\begin{minipage}{\dimexpr0.75\linewidth-\fboxrule-\fboxsep}
		\begin{align*}
			\num{pow} \num{2} = \enskip &\Bigl( \lambda \col{n}fz \ . \ \col{n} \ ( \colll{\lambda gx . g (gx)}) \ f z \Bigr) \, \col{\Bigl( \lambda xy \ .  \ x(xy)) \Bigr)} \\
			%
			\Rightarrow^{\beta} \enskip &\Bigl( \lambda fz \ . \ \Bigl( \lambda \col{x}y \ .  \ \col{x}(\col{x}y)) \Bigr) \ \col{\Bigl( \lambda gx . g (gx) \Bigr)} \ f \ z \Bigr) \\
			%
			\Rightarrow^{\beta} \enskip &\Bigl( \lambda fz \ . \ \biggl( \lambda y \ . \  \Bigl( \lambda \col{g}x . \col{g} (\col{g}x) \Bigr) \, \col{\Bigl( (\lambda gx . g (gx) ) \ y \Bigr)} \biggr) \  f \ z \Bigr) \\
			%
			\Rightarrow^{\beta} \enskip &\Bigl( \lambda fz \ . \ \biggl( \lambda y \ . \  \Bigl( \coll{\lambda x} . \Bigl( (\lambda gx . g (gx) ) \ y \Bigr) \, \Bigl( \bigl( (\lambda gx . g (gx) ) \ y \bigr) \ x \Bigr) \, \Bigr) \,  \biggr) \  f \ z \Bigr) \\
			%
			\Rightarrow^{\beta} \enskip &\Bigl( \lambda fz \ . \ \biggl( \lambda y \coll{x} \ . \Bigl( (\lambda \col{g}x . \col{g} (\col{g}x) ) \ \col{y} \Bigr) \, \Bigl( \bigl( (\lambda \col{g}x . \col{g} (\col{g}x) ) \ \col{y} \bigr) \ x \Bigr) \,  \biggr) \  f \ z \Bigr) \\
			%
			\Rightarrow^{\beta} \enskip &\Bigl( \lambda fz \ . \ \biggl( \lambda y x \ . \Bigl( \lambda x . y (yx)  \Bigr) \, \Bigl( (\lambda \col{x} . y (y\col{x}) ) \ \col{x} \Bigr) \,  \biggr) \  f \ z \Bigr) \\
			%
			\Rightarrow^{\beta} \enskip &\Bigl( \lambda fz \ . \ \biggl( \lambda y x \ . \Bigl( \lambda \col{x} . y (y\col{x})  \Bigr) \, \col{\Bigl( y (yx) \Bigr)} \,  \biggr) \  f \ z \Bigr) \\
			%
			\Rightarrow^{\beta} \enskip &\Bigl( \lambda fz \ . \ \biggl( \lambda \col{y} x \ . \col{y} \ \Bigl(\col{y} \ \bigl( \col{y} \ (\col{y}x) \bigr)\Big) \  \biggr) \  \col{f} \ z \Bigr) \\
			%
			\Rightarrow^{\beta} \enskip &\Bigl( \lambda fz \ . \ \biggl( \lambda \col{x} \ . f \ \Bigl(f \ \bigl( f \ (f \col{x}) \bigr)\Big) \  \biggr) \ \col{z} \Bigr) \qquad
			%
			\Rightarrow^{\beta} \enskip \Bigl( \lambda fz \ . \ f \ (f \ ( f \ (f z) \ ) \ ) \  )  \Bigr) = \num{4} \\
		\end{align*}
	\end{minipage}
	\pause
	\begin{minipage}{\dimexpr0.25\linewidth-\fboxrule-\fboxsep}
		\small \centering
		\emph{Teil (b)} \\[-\baselineskip]
		\begin{equation*} f(n) = s^n \end{equation*}
		
		\bigskip
		\pause
		
		\emph{Teil (c)} \\[-\baselineskip]
		\begin{equation*} g(n,m) = m^n \end{equation*}
		\footnotesize
		\begin{equation*} 
			\num{pow'} = \Bigl( \lambda n \colll{m} f z . n \colll{m} f z \Big) 
		\end{equation*}
		
	\end{minipage}
		
	\end{frame}

	\begin{frame} \frametitle{Fixpunktkombinator und Rekursion}
	\small
		\begin{itemize}
			\item Ein $t \in \Sigma(\lambda)$ heißt \emph{geschlossener Term}, falls $FV(t) = \emptyset$. Ein geschlossender Term heißt auch \emph{Kombinator}.
			\medskip \pause
			\item \emph{Fixpunktkombinator.} $\num{Y} = \Bigl( \lambda z . \ \bigl( \lambda u . z (uu) \bigr) \ \bigl( \lambda u. z (uu) \bigr) \Bigr) \in \lambda(\emptyset)$
			\pause
			\item Der Fixpunktkombinator ermöglicht Rekursion.
			\pause
			\item weitere definierte $\lambda$-Terme (siehe Skript S. 198f.):
			\begin{align*}
				\num{true} &= (\lambda xy. x) & \num{false} &= (\lambda xy. y) \\
				\num{succ} &= (\lambda z . (\lambda xy . x (zxy))) &  \num{pred}\num{0} &\Rightarrow^\ast \num{0} \\
				\num{succ}\num{n} &\Rightarrow^\ast \num{n+1} & 
				\num{pred}\num{n} &\Rightarrow^\ast \num{n-1} \\
			\end{align*}
			\begin{equation*}
				\num{ite} \ s \ s_1 \ s_2 \Rightarrow^\ast \begin{cases}
				s_1 & \text{wenn } s \Rightarrow^\ast \num{true} \\
				s_2 & \text{sonst}
				\end{cases}
			\end{equation*}
		\end{itemize}
	\end{frame}

	\begin{frame} \frametitle{Aufgabe 3 -- Teil (b)}
	\small
		\begin{equation*}
			\num{F} = \Biggl( \lambda f xyz \ . \ \num{ite} \ \Bigl(\num{iszero} \ (\num{sub} x y )\Bigr) \ \Bigl( \num{add} y z \Bigr) \ \Bigl( \num{succ} \ ( f \ (\num{pred} x) \ (\num{succ} y) \ ( \num{mult} \num{2} z )) \Bigr) \Biggr)
		\end{equation*}
		
		\bigskip 
		\pause
		
		\emph{Nebenrechnung:} Zeige die Wirkung des Fixpunktkombinators.
		\pause
		\begin{align*}
			\num{Y} \num{F}
			= \enskip &\Bigl( \lambda z . \ \bigl( \lambda u . z (uu) \bigr) \ \bigl( \lambda u. z (uu) \bigr) \Bigr) \ \num{F} \\
			\Rightarrow^{\beta} \quad &\Bigl( \lambda u . \num{F} (uu) \Bigr) \ \Bigl( \lambda u. \num{F} (uu) \Bigr) \qquad =: \num{Y_F} \\
			\Rightarrow^{\beta} \quad &\num{F} \num{Y_F}
		\end{align*}
	\end{frame}

	\begin{frame} \frametitle{Aufgabe 3 -- Teil (b)}
	\small
		\begin{align*}
			\num{Y} \num{F} \num{6} \num{5} \num{3} &\Rightarrow^\ast \num{F} \num{Y_F} \num{6} \num{5} \num{3} \\
			&\Rightarrow^\ast \num{ite} \ 
			(\underbrace{\num{iszero} \ (\num{sub}\num{6}\num{5})}_{\Rightarrow^\ast \num{false}}) \ 
			( \dots ) \\
			& \phantom{\Rightarrow^\ast \num{ite}} \ (\num{succ} (\num{Y_F}(\underbrace{\num{pred}\num{6}}_{\Rightarrow^\ast \num{5}})(\underbrace{\num{succ}\num{5}}_{\Rightarrow^\ast \num{6}})(\underbrace{\num{mult}\num{2}\num{3}}_{\Rightarrow^\ast \num{6}}))) \\
			&\Rightarrow^\ast \num{succ} \  ( \ \num{Y_F}\num{5}\num{6}\num{6} \ ) \\
			&\Rightarrow^\ast \num{succ} \ ( \ \num{F} \num{Y_F} \num{5}\num{6}\num{6} \ ) \\
			&\Rightarrow^\ast \num{succ} \ ( \ \num{ite} \  (\underbrace{\num{iszero}(\num{sub}\num{5}\num{6})}_{\Rightarrow^\ast \num{true}}) \ (\underbrace{\num{add}\num{6}\num{6}}_{\Rightarrow^\ast \num{12}})  \ (\dots) \ ) \\
			&\Rightarrow^\ast \num{succ} \ \num{12} \\
			&\Rightarrow^\ast \num{13}
		\end{align*}
	\end{frame}
	
	\begin{frame} \frametitle{Aufgabe 3 -- Teil (c)}
	\small
		\begin{alignat*}{3}
			\num{G} = \enskip & \Biggl( \lambda g xy \ . \ \Bigl( &&\num{ite} \ && \bigl( \num{iszero} \ x \bigr) \\
			&&&&& \bigl( \num{mult} \ \num{2} \ (\num{succ} y) \ \bigr) \\
			&&&&& \Bigl( 
			\begin{alignedat}[t]{1}
				\num{ite} \ &(\num{iszero} \ y) \\
				& \bigl( \num{mult} \ \num{2} \ (\num{succ} x) \bigr) \\
				& \bigl( \num{add} \ \num{4} \enskip g \ (\num{pred} x \ \num{pred} y)  \ \bigr) \\				
			\end{alignedat} \\
			&&&&& \Bigr) \\
			&&& \Bigr) \\
			& \Biggr)
		\end{alignat*}
	\end{frame}


\end{document}