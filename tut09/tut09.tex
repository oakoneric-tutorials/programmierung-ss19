\documentclass[aspectratio=1610,onlymath, ngerman]{beamer}
% \documentclass[aspectratio=1610,onlymath,handout]{beamer}
\usepackage[ngerman]{babel}
\usepackage[utf8]{inputenc}
\usepackage[T1]{fontenc}

\usetheme[nosectionnum,pagenum,sansmath, cd2018, nodin]{tud}

%\usefonttheme[onlymath]{serif}
\usepackage{opensans}
\usepackage{stmaryrd}
\usepackage[normalem]{ulem} % sout command
\usepackage{txfonts}
\DeclareMathAlphabet{\mathsc}{OT1}{cmr}{m}{sc}

\usepackage{listings}
\lstset{language=Haskell,basicstyle=\ttfamily}
\usepackage{verbatim}
\usepackage{bold-extra}

\setbeamerfont{title}{size=\Huge, family=\bfseries\fosfamily}
\setbeamerfont{frametitle}{size=\LARGE, family=\bfseries\fosfamily}

\setbeamerfont{normal text}{size=\normalsize}
\setbeamerfont{itemize/enumerate body}{}
\setbeamerfont{itemize/enumerate subbody}{size=\small}
\setbeamerfont{itemize/enumerate subsubbody}{size=\footnotesize}

\usepackage{tudscrcolor}
\usepackage{environ}
\usepackage{tikz}
\usetikzlibrary{arrows,positioning,decorations.pathreplacing}
% Inspired by http://www.texample.net/tikz/examples/hand-drawn-lines/
\usetikzlibrary{decorations.pathmorphing}
\pgfdeclaredecoration{penciline}{initial}{
    \state{initial}[width=+\pgfdecoratedinputsegmentremainingdistance,
    auto corner on length=1mm,]{
        \pgfpathcurveto%
        {% From
            \pgfqpoint{\pgfdecoratedinputsegmentremainingdistance}
            {\pgfdecorationsegmentamplitude}
        }
        {%  Control 1
            \pgfmathrand
            \pgfpointadd{\pgfqpoint{\pgfdecoratedinputsegmentremainingdistance}{0pt}}
            {\pgfqpoint{-\pgfdecorationsegmentaspect
                    \pgfdecoratedinputsegmentremainingdistance}%
                {\pgfmathresult\pgfdecorationsegmentamplitude}
            }
        }
        {%TO
            \pgfpointadd{\pgfpointdecoratedinputsegmentlast}{\pgfpoint{1pt}{1pt}}
        }
    }
    \state{final}{}
}
\tikzset{handdrawn/.style={decorate,decoration=penciline}}
\tikzset{every shadow/.style={fill=none,shadow xshift=0pt,shadow yshift=0pt}}

\NewEnviron{doodlebox}[2]{%
    \begin{tikzpicture}[decoration=penciline, decorate]%
    \pgfmathsetseed{1237}%
    \node (n1) [decorate,draw=#1, fill=#2,thick,align=justify, text width=0.97\textwidth, inner ysep=2mm, inner xsep=2mm] at (0,0) {\BODY};%
    \end{tikzpicture}%
}
\NewEnviron{doodle}[1]{%
    \begin{tikzpicture}[decoration=penciline, decorate]%
    \pgfmathsetseed{1237}%
    \node (n1) [decorate,draw=#1, fill=#1!10,thick,align=justify, text width=0.97\textwidth, inner ysep=2mm, inner xsep=2mm] at (0,0) {\BODY};%
    \end{tikzpicture}%
}

\newcommand{\defineTitle}[3]{%
    \newcommand{\lectureindex}{#1}%
    \title{Programmierung}%
    \subtitle{Übung #1: #2}%
    \author{Eric Kunze \\ \url{eric.kunze@mailbox.tu-dresden.de} }%
    \date{#3}%
    \datecity{TU Dresden}%
}

%%%%%%%%%%%%%%%%%%%%%%%%%%%%%%%%%%%%%%%%%%%%%%%%%%%%%%%%%%%%%%%%%%%%%%%%%%%%
%%%%%%%%%%%%%%%%%%%%%%%%%%%%%%%%%%%%%%%%%%%%%%%%%%%%%%%%%%%%%%%%%%%%%%%%%%%%

\defineTitle{9}{Logik-Programmierung in Prolog}{07.~Juni~2019}

\renewcommand{\emph}[1]{\textbf{#1}}
\newcommand{\coloremph}[1]{\textcolor{cdpurple}{#1}}
\newcommand{\col}[1]{\textcolor{cdpurple}{\boldsymbol{#1}}}
\newcommand{\coll}[1]{\textcolor{cddarkgreen}{\boldsymbol{#1}}}
\newcommand{\colll}[1]{\textcolor{cdorange}{\boldsymbol{#1}}}

\DeclareMathSymbol{*}{\mathbin}{symbols}{"01}

\renewcommand*{\headerinfo}{\color{cdgray}\textbf{Github:} \url{https://github.com/oakoneric/programmierung-ss19}}
\arraycolsep2pt

\usepackage{aligned-overset}
\usepackage{array}
\chardef\_=`_

\newcommand{\cw}[1]{\texttt{#1}}
\newcommand{\step}[2][]{\ensuremath{\overset{{#1} (\text{#2})}{=}}}
\newcommand*{\astep}[2][]{\ensuremath{\overset{{#1} (\text{#2})}&{=}}}

\newcommand{\num}[1]{\ensuremath{\langle #1 \rangle}}


\begin{document}
    \maketitle
    
	\begin{frame} \frametitle{Aufgabe 1 -- Teil (a)}
	\small
		\begin{ttfamily}
			\begin{tabbing}
				1 \quad \= nat(0). \\
				2 \> nat(s(X)) :- nat(X). \\[9pt]
				3 \> sum(0,Y,Y) :- nat(Y). \\
				4 \> sum(s(X), Y, s(S)) :- sum(X,Y,S). \\[9pt] \pause
				5 \> lt(0, s(M)) \pause :- nat(M). \\ \pause
				6 \> lt(s(N),s(M)) \pause :- lt(N,M). \\[9pt] \pause
				7 \> div(0,M,0) \pause :- lt(0,M). \\ \pause
				8 \> div(N,M,0) \pause :- lt(N,M). \\ \pause
				9 \> div(N,M,s(Q)) \pause :- lt(0,M), sum(M,V,N), div(V,M,Q).
			\end{tabbing}
		\end{ttfamily}	
		
	\end{frame}
    
    \begin{frame} \frametitle{Aufgabe 1 -- Teil (b)}
	    \begin{alignat*}{2}
		    &\texttt{?- div(\num{3}, \num{2}, \num{1})} \\
		    &\texttt{?- lt(\num{0}, \num{2}) , sum(\num{2}, V1, \num{3}) , div(V1, \num{2}, \num{0})} \qquad && \texttt{\% 9 } \\
		    &\texttt{?- nat(\num{1}) , sum(\num{2}, V1, \num{3}) , div(V1, \num{2}, \num{0})}  && \texttt{\% 5} \\
		    &\texttt{?- nat(\num{0}) , sum(\num{2}, V1, \num{3}) , div(V1, \num{2}, \num{0})} && \texttt{\% 2} \\
		    &\texttt{?- sum(\num{2}, V1, \num{3}) , div(V1, \num{2}, \num{0}).} && \texttt{\% 1} \\
		    &\texttt{?-* sum(\num{0}, V1, \num{1}) , div(V1, \num{2}, \num{0}).} && \texttt{\% 4} \\
		    \texttt{\{V1=\num{1}\}} \quad &\texttt{?- nat(\num{1}) , div(\num{1}, \num{2}, \num{0}).} && \texttt{\% 3} \\
		    &\texttt{?- nat(\num{0}) , div(\num{1}, \num{2}, \num{0}).} \quad && \texttt{\% 2} \\
		    &\texttt{?- div(\num{1}, \num{2}, \num{0}).} && \texttt{\% 1} \\
		    &\texttt{?- lt(\num{1}, \num{2}).} &&\texttt{\% 8} \\
		    &\texttt{?- lt(\num{0}, \num{1}).} &&\texttt{\% 6} \\
		    &\texttt{?- nat(\num{0}).} &&\texttt{\% 5} \\
		    &\texttt{?- .} &&\texttt{\% 1} \\
	    \end{alignat*}
	\end{frame}
    
    \begin{frame} \frametitle{Aufgabe 2}
    \small
    	\setlength{\tabcolsep}{2pt}
   		\begin{tabular}{>{\ttfamily}l >{\ttfamily}c >{\ttfamily}c >{\ttfamily}c >{\ttfamily}r >{\ttfamily}c >{\ttfamily}l}
	   		eval( & X         &, &X &) & :- & nat(X). \\
	   		eval( & plus(L,R) &, &X &) & :- & eval(L,LE), eval(R,RE), sum(LE,RE,X). \\
	   		eval( & minus(L,R) &, &X &) & :- & eval(L,LE), eval(R,RE), sum(RE, X, LE).
   		\end{tabular}
	\end{frame}
    
    \begin{frame} \frametitle{Aufgabe 3}
    \small
	    \setlength{\tabcolsep}{2pt}
	    \begin{tabular}{>{\ttfamily\scriptsize}l >{\ttfamily}l >{\ttfamily}c >{\ttfamily}c >{\ttfamily}c >{\ttfamily}r >{\ttfamily}c >{\ttfamily}l}
	    	1 & subt( & X         &, &X &).&& \\
	    	2 & subt( & S1 &, &s(\_, T2) &) & :- & subt(S1,T2). \\
	    	3 & subt( & S1 &, &s(T1, \_) &) & :- & subt(S1,T1). \\
	    \end{tabular}
    
    	\bigskip \pause
    	
    	\emph{Teil (a).}
    	
    \footnotesize
    
    	\begin{tabular}{>{\ttfamily}r l >{\ttfamily}l >{\ttfamily}l}
    		& ?- & subt(s(X,Y), s(s(a,b), s(b,a))). \\
    		\{X = s(a,b), Y=s(b,a)\} & ?- & . & \% 1 \\[9pt] %
    		%
    		& ?- & subt(s(X,Y), s(s(a,b), s(b,a))). \\
    		& ?- & subt(s(X,Y), s(b,a)). & \% 2 \\
    		\{X = b, Y=a\} & ?- & . & \% 1 \\[9pt]
    		%
	    	& ?- & subt(s(X,Y), s(s(a,b), s(b,a))). \\
	    	& ?- & subt(s(X,Y), s(a,b)). & \% 3 \\
	    	\{X = a, Y=b \} & ?- & . & \% 1
	    \end{tabular}
	\end{frame}

	\begin{frame} \frametitle{Aufgabe 3}
	\small
		
		\emph{Teil (b).}
		
		\begin{tabular}{>{\ttfamily}r l >{\ttfamily}l >{\ttfamily}l}
			& ?- & subt(s(a,a), X). \\
			\{X = s(a,a)\} &?- & . & \% 1 \\[9pt] %
			%
			& ?- & subt(s(a,a), X). \\
			\{X = s(\_ , X1) \} & ?- & subt(s(a,a), X1). & \% 2 \\
			\{X1 = s(a,a)\} & ?- & . & \% 1 $\qquad \Rightarrow$ X = s(a,s(a,a))\\[9pt]
			%
			& ?- & subt(s(a,a), X). \\
			\{X = s(X2, \_)\}& ?- & subt(s(a,a), X2). & \% 3 \\
			\{X2 = s(a,a) \} & ?- & . & \% 1 $\qquad \Rightarrow$ X = s(s(a,a),c)
		\end{tabular}
	\end{frame}
\end{document}