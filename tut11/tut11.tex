\documentclass[aspectratio=1610,onlymath, ngerman]{beamer}
% \documentclass[aspectratio=1610,onlymath,handout]{beamer}
\usepackage[ngerman]{babel}
\usepackage[utf8]{inputenc}
\usepackage[T1]{fontenc}

\usetheme[nosectionnum,pagenum,sansmath, cd2018, nodin]{tud}

%\usefonttheme[onlymath]{serif}
\usepackage{opensans}
\usepackage{stmaryrd}
\usepackage[normalem]{ulem} % sout command
\usepackage{txfonts}
\DeclareMathAlphabet{\mathsc}{OT1}{cmr}{m}{sc}

\usepackage{listings}
\lstset{language=Haskell,basicstyle=\ttfamily}
\usepackage{verbatim}
\usepackage{bold-extra}

\setbeamerfont{title}{size=\Huge, family=\bfseries\fosfamily}
\setbeamerfont{frametitle}{size=\LARGE, family=\bfseries\fosfamily}

\setbeamerfont{normal text}{size=\normalsize}
\setbeamerfont{itemize/enumerate body}{}
\setbeamerfont{itemize/enumerate subbody}{size=\small}
\setbeamerfont{itemize/enumerate subsubbody}{size=\footnotesize}

\usepackage{tudscrcolor}
\usepackage{environ}
\usepackage{tikz}
\usetikzlibrary{arrows,positioning,decorations.pathreplacing}
% Inspired by http://www.texample.net/tikz/examples/hand-drawn-lines/
\usetikzlibrary{decorations.pathmorphing}
\pgfdeclaredecoration{penciline}{initial}{
    \state{initial}[width=+\pgfdecoratedinputsegmentremainingdistance,
    auto corner on length=1mm,]{
        \pgfpathcurveto%
        {% From
            \pgfqpoint{\pgfdecoratedinputsegmentremainingdistance}
            {\pgfdecorationsegmentamplitude}
        }
        {%  Control 1
            \pgfmathrand
            \pgfpointadd{\pgfqpoint{\pgfdecoratedinputsegmentremainingdistance}{0pt}}
            {\pgfqpoint{-\pgfdecorationsegmentaspect
                    \pgfdecoratedinputsegmentremainingdistance}%
                {\pgfmathresult\pgfdecorationsegmentamplitude}
            }
        }
        {%TO
            \pgfpointadd{\pgfpointdecoratedinputsegmentlast}{\pgfpoint{1pt}{1pt}}
        }
    }
    \state{final}{}
}
\tikzset{handdrawn/.style={decorate,decoration=penciline}}
\tikzset{every shadow/.style={fill=none,shadow xshift=0pt,shadow yshift=0pt}}

\NewEnviron{doodlebox}[2]{%
    \begin{tikzpicture}[decoration=penciline, decorate]%
    \pgfmathsetseed{1237}%
    \node (n1) [decorate,draw=#1, fill=#2,thick,align=justify, text width=0.97\textwidth, inner ysep=2mm, inner xsep=2mm] at (0,0) {\BODY};%
    \end{tikzpicture}%
}
\NewEnviron{doodle}[1]{%
    \begin{tikzpicture}[decoration=penciline, decorate]%
    \pgfmathsetseed{1237}%
    \node (n1) [decorate,draw=#1, fill=#1!10,thick,align=justify, text width=0.97\textwidth, inner ysep=2mm, inner xsep=2mm] at (0,0) {\BODY};%
    \end{tikzpicture}%
}

\newcommand{\defineTitle}[3]{%
    \newcommand{\lectureindex}{#1}%
    \title{Programmierung}%
    \subtitle{Übung #1: #2}%
    \author{Eric Kunze \\ \url{eric.kunze@mailbox.tu-dresden.de} }%
    \date{#3}%
    \datecity{TU Dresden}%
}

%%%%%%%%%%%%%%%%%%%%%%%%%%%%%%%%%%%%%%%%%%%%%%%%%%%%%%%%%%%%%%%%%%%%%%%%%%%%
%%%%%%%%%%%%%%%%%%%%%%%%%%%%%%%%%%%%%%%%%%%%%%%%%%%%%%%%%%%%%%%%%%%%%%%%%%%%

\defineTitle{11}{$C_1$ und $AM_1$}{28.~Juni~2019}

\usepackage[inline]{enumitem}

\renewcommand{\emph}[1]{\textbf{#1}}
\newcommand{\coloremph}[1]{\textcolor{cdpurple}{#1}}
\newcommand{\col}[1]{\textcolor{cdpurple}{\boldsymbol{#1}}}
\newcommand{\coll}[1]{\textcolor{cddarkgreen}{\boldsymbol{#1}}}
\newcommand{\colll}[1]{\textcolor{cdorange}{\boldsymbol{#1}}}

\DeclareMathSymbol{*}{\mathbin}{symbols}{"01}

\renewcommand*{\headerinfo}{\color{cdgray}\textbf{Github:} \url{https://github.com/oakoneric/programmierung-ss19}}
\arraycolsep2pt

\usepackage{aligned-overset}
\usepackage{array}
\chardef\_=`_

\renewcommand{\epsilon}{\varepsilon}
\newcommand{\labelitemi}{$\blacktriangleright$}
\newcommand{\labelitemii}{$\vartriangleright$}
\usepackage{tabu}
\newcommand*\head{\rowfont{\bfseries}}
\newcommand*{\tw}{\rowfont{\ttfamily}}

\begin{document}
    \maketitle
    
    \begin{frame} \frametitle{$C_1$ und $AM_1$}
    \small
    	\begin{itemize}
    		\item \emph{bisher:} Implementierung von $C_0$ mit $AM_0$
    		\item \emph{jetzt:} Erweiterung auf $C_1$ mit $AM_1$ \pause
    			\begin{itemize}
    				\item Erweiterung um Funktionen ohne Rückgabewert
    				\item Einschränkungen von $C_0$ bleiben erhalten
    			\end{itemize}
    		\pause
    		\item \emph{Implementierung} durch
    			\begin{itemize}
    				\item Syntax von $C_1$
    				\item Befehle und Semantik einer abstrakten Maschine $AM_1$
    				\item Übersetzer $C_1 \leftrightarrow AM_1$
    			\end{itemize}
    	\end{itemize}
    \end{frame}

	\begin{frame} \frametitle{Befehlssemantik der $AM_1$}
	\small
		\begin{minipage}{\dimexpr0.5\linewidth-\fboxrule-\fboxsep}
			$b \in \{ \text{global}, \text{lokal}\}$ \\
			$r$ \dots aktueller REF
		\end{minipage}
		\begin{minipage}{\dimexpr0.5\linewidth-\fboxrule-\fboxsep}
			\begin{equation*}
				adr(r,b,o) = \begin{cases}
				r+o & \text{wenn } b = \text{lokal} \\
				o & \text{wenn } b = \text{global}
				\end{cases}
			\end{equation*}
		\end{minipage}
		
		\smallskip
		
		\begin{tabular}{p{2cm} p{\dimexpr\linewidth-\fboxrule-\fboxsep-2cm}}
			\emph{Befehl} & \emph{Auswirkungen}  \\ \hline
			LOAD($b$,$o$) & Lädt den Inhalt von Adresse $adr(r,b,o)$ auf den Datenkeller, inkrementiere Befehlszähler \\
			STORE($b$,$o$) & Speichere oberstes Datenkellerelement an $adr(r,b,o)$, inkrementiere Befehlszähler \\
			WRITE($b$,$o$) & Schreibe Inhalt an Adresse $adr(r,b,o)$ auf das
			Ausgabeband, inkrementiere Befehlszähler \\
			READ($b$,$o$) & Lies oberstes Element vom Eingabeband, speichere
			an Adresse $adr(r,b,o)$, inkrementiere Befehlszähler \\
		\end{tabular}
	\end{frame}

	\begin{frame} \frametitle{Befehlssemantik der $AM_1$}
%	\small		
		\begin{tabular}{p{2cm} p{\dimexpr\linewidth-\fboxrule-\fboxsep-2cm}}
			\emph{Befehl} & \emph{Auswirkungen}  \\ \hline
			LOADI($o$) & Ermittle Wert ($=b$) an Adresse $r+o$, Lade Inhalt
			von Adresse $b$ auf Datenkeller, inkrementiere Befehlszähler \\
			STOREI($o$) & Ermittle Wert ($=b$) an Adresse $r+o$, nimm
			oberstes Datenkellerelement, speichere dieses an
			Adresse $b$, inkrementiere Befehlszähler \\
			WRITEI($o$) & Ermittle Wert ($=b$) an Adresse $r+o$, schreibe den
			Inhalt an Adresse $b$ auf Ausgabeband, inkrementiere Befehlszähler \\
			READI($o$) & Ermittle Wert ($=b$) an Adresse $r+o$, lies das oberste
			Element vom Eingabeband, speichere es an Adresse $b$ , inkrementiere Befehlszähler \\
			LOADA($b$,$o$) & Lege $adr(r,b,o)$ auf Datenkeller, inkrementiere Befehlszähler \\			
			PUSH & oberstes Element vom Datenkeller auf Laufzeitkeller, Befehlszähler inkrementieren \\
			CALL $adr$ & Befehlszählerwert inkrementieren und auf LZK legen, Befehlszähler auf $adr$ setzen, REF auf LZK legen, REF auf Länge des LZK ändern \\
			INIT $n$ & $n$-mal $0$ auf den Laufzeitkeller legen \\
			RET $n$ & im LZK alles nach REF-Zeiger löschen, oberstes Element des LZK als REF setzen, oberstes Element des LZK als Befehlszähler setzen, $n$ Elemente von LZK löschen
		\end{tabular}
	\end{frame}


	\begin{frame} \frametitle{Aufgabe 1 -- Teil (a)}
	\small
	
	\begin{ttfamily}
		\emph{while} (*p > i) \{ f(p); i = i + 1; \} \\
		p = \&i;
	\end{ttfamily}

	\begin{equation*}
		tab_{g+\text{lDecl}} = \{ \texttt{f}/(\text{proc}, 1), \texttt{g}/(\text{proc}, 2), \texttt{i}/(\text{var}, \text{lokal}, 1), \texttt{p}/(\text{var-ref}, -2) \}
	\end{equation*}
	
	\bigskip 
	\pause
	\emph{Lösung.}
	
	\medskip 
	
	\begin{tabular}{>{\ttfamily}r >{\ttfamily}l >{\ttfamily}l >{\ttfamily}l >{\ttfamily}l >{\ttfamily}l}
		2.2.1 & LOADI(-2) ;  LOAD(lokal,1) ;  GT ;  JMC 2.2.2 ; \\
		& LOAD(lokal,-2)  ;  PUSH          ;  CALL 1 ; \\
		& LOAD(lokal,1)   ;  LIT 1         ; ADD ; STORE(lokal,1) ; JMP 2.2.1 ; \medskip \\
		2.2.2 & LOADA(lokal,1) ; STORE(lokal,-2) ; \\
	\end{tabular}

		\begin{minipage}{\dimexpr0.5\linewidth-\fboxrule-\fboxsep}
		\end{minipage}

	\end{frame}

	\begin{frame} \frametitle{Aufgabe 1 -- Teil (b)}
	\small
	
		Gegeben ist folgender $AM_1$-Code: \\
		\bigskip 
		
		\begin{minipage}{\dimexpr0.33\linewidth-\fboxrule-\fboxsep}
			\begin{ttfamily}
				\begin{enumerate}[label=\arabic*:, nolistsep]
					\item INIT 1;
					\item CALL 18;
					\item INIT 0;
					\item LOAD(lokal , -2);
					\item LIT 0;
					\item GT;
					\item JMC 17;
					\item LIT 2;
					\item LOADI ( -3);
				\end{enumerate}
			\end{ttfamily}
		\end{minipage}
		\begin{minipage}{\dimexpr0.33\linewidth-\fboxrule-\fboxsep}
			\begin{ttfamily}
				\begin{enumerate}[label=\arabic*:, nolistsep]
					\setcounter{enumi}{9}
					\item MUL;
					\item STOREI ( -3);
					\item LOAD(lokal , -2);
					\item LIT 1;
					\item SUB;
					\item STORE(lokal , -2);
					\item JMP 4;
					\item RET 2;
					\item INIT 0;
				\end{enumerate}
			\end{ttfamily}
		\end{minipage}
		\begin{minipage}{\dimexpr0.33\linewidth-\fboxrule-\fboxsep}
			\begin{ttfamily}
				\begin{enumerate}[label=\arabic*:, nolistsep]
					\setcounter{enumi}{18}
					\item READ(global ,1);
					\item LOADA(global ,1);
					\item PUSH;
					\item LOAD(global ,1);
					\item PUSH;
					\item CALL 3;
					\item WRITE(global ,1);
					\item JMP 0;

				\end{enumerate}
			\end{ttfamily}
		\end{minipage}
	
		\bigskip
		
		Führen Sie $11$ Schritte der $AM_1$ auf der Konfiguration $\sigma = (22, \epsilon, 1:3:0:1, 3, \epsilon, \epsilon)$ aus.
	\end{frame}
    
    \begin{frame} \frametitle{Aufgabe 1 -- Teil (b)}
    \footnotesize
    	
    	\begin{center}
    		\begin{tabu}{rccrclcccrcrl}
    			\head & BZ && DK && LZK && REF && Inp && Out & \\ \hline 
    			( & 22 &,& $\epsilon$ &,& 1:3:0:1 &,& 3 &,& $\epsilon$ &,& $\epsilon$ & ) \\
    			( & 23 &,& 1 &,& 1:3:0:1 &,& 3 &,& $\epsilon$ &,& $\epsilon$ & ) \\
    			( & 24 &,& $\epsilon$ &,& 1:3:0:1:1 &,& 3 &,& $\epsilon$ &,& $\epsilon$ & ) \\
    			( & 3 &,& $\epsilon$ &,& 1:3:0:1:1:25:3 &,& 7 &,& $\epsilon$ &,& $\epsilon$ & ) \\
    			( & 4 &,& $\epsilon$ &,& 1:3:0:1:1:25:3 &,& 7 &,& $\epsilon$ &,& $\epsilon$ & ) \\
    			( & 5 &,& 1 &,& 1:3:0:1:1:25:3 &,& 7 &,& $\epsilon$ &,& $\epsilon$ & ) \\
    			( & 6 &,& 0:1 &,& 1:3:0:1:1:25:3 &,& 7 &,& $\epsilon$ &,& $\epsilon$ & ) \\
    			( & 7 &,& 1 &,& 1:3:0:1:1:25:3 &,& 7 &,& $\epsilon$ &,& $\epsilon$ & ) \\
    			( & 8 &,& $\epsilon$ &,& 1:3:0:1:1:25:3 &,& 7 &,& $\epsilon$ &,& $\epsilon$ & ) \\
    			( & 9 &,& 2 &,& 1:3:0:1:1:25:3 &,& 7 &,& $\epsilon$ &,& $\epsilon$ & ) \\
    			( & 10 &,& 1:2 &,& 1:3:0:1:1:25:3 &,& 7 &,& $\epsilon$ &,& $\epsilon$ & ) \\
    			( & 11 &,& 2 &,& 1:3:0:1:1:25:3 &,& 7 &,& $\epsilon$ &,& $\epsilon$ & ) \\
    			( & 12 &,& $\epsilon$ &,& 2:3:0:1:1:25:3 &,& 7 &,& $\epsilon$ &,& $\epsilon$ & ) \\
    			( & 13 &,& 1 &,& 2:3:0:1:1:25:3 &,& 7 &,& $\epsilon$ &,& $\epsilon$ & ) \\
    			( & 14 &,& 1:1 &,& 2:3:0:1:1:25:3 &,& 7 &,& $\epsilon$ &,& $\epsilon$ & ) \\
    		\end{tabu}
    	\end{center}
    \end{frame}

	\begin{frame} \frametitle{Aufgabe 2}
	\small
		\begin{ttfamily}
			\emph{\#} \!\!\!\!\!\! \emph{include} <stdio.h> \\ 
			\emph{int} x, y; \\ \medskip
			\emph{void} f(...) {...} \\ \medskip
			\emph{void} g(\emph{int} a, \emph{int} *b) \{ \\
			$\quad$	\emph{int} c; \\
			$\quad$	c = 3; \\
			$\quad$	\emph{if} (c == *b) \emph{while} (a > 0) f(\&a, b); \\
			\} \\ \medskip
			\emph{void} main () \{ ... \} \\
		\end{ttfamily}
	\end{frame}

	\begin{frame} \frametitle{Aufgabe 2}
	\small 
		
		\begin{equation*}
			\begin{aligned}
			lokal-tab_{\texttt{g}} = [&\text{f/(proc,1) , g/(proc,2) , x/(var,global,1) , y/(var,global,2) ,} \\
				&\text{a/(var,lokal,--3) , b/(var-ref,--2) , c/(var,lokal,1)}]
			\end{aligned}
		\end{equation*}
		
		\bigskip 
		\pause
		\emph{Lösung.}
		
		\medskip 
		
		\begin{tabu}{rl}
			\tw &LIT 3 ; STORE(lokal,1); \\
			\tw &LOAD(lokal,1) ; LOADI(-2) ; EQ ; JMC 2.2.1 ; \\
			\tw 2.2.2.1: & LOAD(lokal,-3) ; LIT 0 ; GT ; JMC 2.2.2.2 ; \\
			\tw & LOADA(lokal,-3) ; PUSH ; LOAD(lokal,-2) ; PUSH ; CALL 1 ; \\
			\tw & JMP 2.2.2.1 \\
			\tw 2.2.2.2: & 2.2.1:
		\end{tabu}
	\end{frame}
	
	\begin{frame} \frametitle{Aufgabe 3}
	\small
		Gegeben ist folgender $AM_1$-Code: \\
		\bigskip 
		
		\begin{minipage}{\dimexpr0.33\linewidth-\fboxrule-\fboxsep}
			\begin{ttfamily}
				\begin{enumerate}[label=\arabic*:, nolistsep]
					\item INIT 1;
					\item CALL 1§;
					\item INIT 0;
					\item LOADI(-2);
					\item LIT 2;
					\item GT;
					\item JMC 12;
				\end{enumerate}
			\end{ttfamily}
		\end{minipage}
		\begin{minipage}{\dimexpr0.33\linewidth-\fboxrule-\fboxsep}
			\begin{ttfamily}
				\begin{enumerate}[label=\arabic*:, nolistsep]
					\setcounter{enumi}{7}
					\item LOADI(-2);
					\item LIT 2;
					\item DIV;
					\item STOREI(-2);
					\item RET 1;
					\item INIT 0;
					\item READ(global, 1);
				\end{enumerate}
			\end{ttfamily}
		\end{minipage}
		\begin{minipage}{\dimexpr0.33\linewidth-\fboxrule-\fboxsep}
			\begin{ttfamily}
				\begin{enumerate}[label=\arabic*:, nolistsep]
					\setcounter{enumi}{14}
					\item LOADA(global ,1);
					\item PUSH;
					\item CALL 3;
					\item WRITE(global ,1);
					\item JMP 0;

				\end{enumerate}
			\end{ttfamily}
		\end{minipage}
		
		\bigskip
		
		Gesucht ist das Ablaufprotokoll der $AM_1$ mit der Anfangskonfiguration $\sigma = (14, \epsilon, 0:0:1, 3, 4, \epsilon)$.
	\end{frame}

	\begin{frame} \frametitle{Aufgabe 3}
	\footnotesize
		\vspace{-6pt}
		\begin{center}
			\begin{tabu}{rrcrclcccrcrl}
				\head & BZ && DK && LZK && REF && Inp && Out & \\ \hline
				( & 14 &,& $\epsilon$ &,& 0:0:1 &,& 3 &,& 4 &,& $\epsilon$ & ) \\
				( & 15 &,& $\epsilon$ &,& 4:0:1 &,& 3 &,& $\epsilon$ &,& $\epsilon$ & ) \\
				( & 16 &,& 1 &,& 4:0:1 &,& 3 &,& $\epsilon$ &,& $\epsilon$ & ) \\
				( & 17 &,& $\epsilon$ &,& 4:0:1:1 &,& 3 &,& $\epsilon$ &,& $\epsilon$ & ) \\
				( & 3 &,& $\epsilon$ &,& 4:0:1:1:18:3 &,& 6 &,& $\epsilon$ &,& $\epsilon$ & ) \\
				( & 4 &,& $\epsilon$ &,& 4:0:1:1:18:3 &,& 6 &,& $\epsilon$ &,& $\epsilon$ & ) \\
				( & 5 &,& 4 &,& 4:0:1:1:18:3 &,& 6 &,& $\epsilon$ &,& $\epsilon$ & ) \\
				( & 6 &,& 2:4 &,& 4:0:1:1:18:3 &,& 6 &,& $\epsilon$ &,& $\epsilon$ & ) \\
				( & 7 &,& 1 &,& 4:0:1:1:18:3 &,& 6 &,& $\epsilon$ &,& $\epsilon$ & ) \\
				( & 8 &,& $\epsilon$ &,& 4:0:1:1:18:3 &,& 6 &,& $\epsilon$ &,& $\epsilon$ & ) \\
				( & 9 &,& 4 &,& 4:0:1:1:18:3 &,& 6 &,& $\epsilon$ &,& $\epsilon$ & ) \\
				( & 10 &,& 2:4 &,& 4:0:1:1:18:3 &,& 6 &,& $\epsilon$ &,& $\epsilon$ & ) \\
				( & 11 &,& 2 &,& 4:0:1:1:18:3 &,& 6 &,& $\epsilon$ &,& $\epsilon$ & ) \\
				( & 12 &,& $\epsilon$ &,& 2:0:1:1:18:3 &,& 6 &,& $\epsilon$ &,& $\epsilon$ & ) \\
				( & 18 &,& $\epsilon$ &,& 2:0:1 &,& 3 &,& $\epsilon$ &,& $\epsilon$ & ) \\
				( & 19 &,& $\epsilon$ &,& 2:0:1 &,& 3 &,& $\epsilon$ &,& 2 & ) \\
			\end{tabu}
		\end{center}
	\end{frame}
\end{document}