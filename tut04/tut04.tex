\documentclass[aspectratio=1610,onlymath, ngerman]{beamer}
% \documentclass[aspectratio=1610,onlymath,handout]{beamer}
\usepackage[ngerman]{babel}
\usepackage[utf8]{inputenc}
\usepackage[T1]{fontenc}

\usetheme[nosectionnum,pagenum,sansmath, cd2018, nodin]{tud}

%\usefonttheme[onlymath]{serif}
\usepackage{opensans}
\usepackage{stmaryrd}
\usepackage[normalem]{ulem} % sout command
\usepackage{txfonts}
\DeclareMathAlphabet{\mathsc}{OT1}{cmr}{m}{sc}

\usepackage{listings}
\lstset{language=Haskell,basicstyle=\ttfamily}
\usepackage{verbatim}
\usepackage{bold-extra}

\setbeamerfont{title}{size=\Huge, family=\bfseries\fosfamily}
\setbeamerfont{frametitle}{size=\LARGE, family=\bfseries\fosfamily}

\setbeamerfont{normal text}{size=\normalsize}
\setbeamerfont{itemize/enumerate body}{}
\setbeamerfont{itemize/enumerate subbody}{size=\small}
\setbeamerfont{itemize/enumerate subsubbody}{size=\footnotesize}

\usepackage{tudscrcolor}
\usepackage{environ}
\usepackage{tikz}
\usetikzlibrary{arrows,positioning,decorations.pathreplacing}
% Inspired by http://www.texample.net/tikz/examples/hand-drawn-lines/
\usetikzlibrary{decorations.pathmorphing}
\pgfdeclaredecoration{penciline}{initial}{
    \state{initial}[width=+\pgfdecoratedinputsegmentremainingdistance,
    auto corner on length=1mm,]{
        \pgfpathcurveto%
        {% From
            \pgfqpoint{\pgfdecoratedinputsegmentremainingdistance}
            {\pgfdecorationsegmentamplitude}
        }
        {%  Control 1
            \pgfmathrand
            \pgfpointadd{\pgfqpoint{\pgfdecoratedinputsegmentremainingdistance}{0pt}}
            {\pgfqpoint{-\pgfdecorationsegmentaspect
                    \pgfdecoratedinputsegmentremainingdistance}%
                {\pgfmathresult\pgfdecorationsegmentamplitude}
            }
        }
        {%TO
            \pgfpointadd{\pgfpointdecoratedinputsegmentlast}{\pgfpoint{1pt}{1pt}}
        }
    }
    \state{final}{}
}
\tikzset{handdrawn/.style={decorate,decoration=penciline}}
\tikzset{every shadow/.style={fill=none,shadow xshift=0pt,shadow yshift=0pt}}

\NewEnviron{doodlebox}[2]{%
    \begin{tikzpicture}[decoration=penciline, decorate]%
    \pgfmathsetseed{1237}%
    \node (n1) [decorate,draw=#1, fill=#2,thick,align=justify, text width=0.97\textwidth, inner ysep=2mm, inner xsep=2mm] at (0,0) {\BODY};%
    \end{tikzpicture}%
}
\NewEnviron{doodle}[1]{%
    \begin{tikzpicture}[decoration=penciline, decorate]%
    \pgfmathsetseed{1237}%
    \node (n1) [decorate,draw=#1, fill=#1!10,thick,align=justify, text width=0.97\textwidth, inner ysep=2mm, inner xsep=2mm] at (0,0) {\BODY};%
    \end{tikzpicture}%
}

\newcommand{\defineTitle}[3]{%
    \newcommand{\lectureindex}{#1}%
    \title{Programmierung}%
    \subtitle{Übung #1: #2}%
    \author{Eric Kunze \\ \url{eric.kunze@mailbox.tu-dresden.de} }%
    \date{#3}%
    \datecity{TU Dresden}%
}

\defineTitle{4}{Haskell -- Typpolymorphie \& Unifikation}{\today}
\renewcommand{\emph}[1]{\textbf{#1}}
\newcommand{\coloremph}[1]{\textcolor{cdpurple}{#1}}
\newcommand{\col}[1]{\textcolor{cdpurple}{\boldsymbol{#1}}}
\DeclareMathSymbol{*}{\mathbin}{symbols}{"01}
    
\begin{document}
    \maketitle
    
    \begin{frame}\frametitle{Unifikation}
	%\begin{minipage}{\dimexpr0.5\linewidth-\fboxrule-\fboxsep}
%		\small
%		\begin{doodle}{cdorange}
%			\emph{Verlegung an Himmelfahrt:} \\
%			von Do, 30.05., 1. DS auf 
%			\emph{Mittwoch, 29.05., 1.DS, APB E001}
%		\end{doodle}
%	
%	\bigskip
	
	%\end{minipage}
%	\pause
	\small
	\begin{minipage}{\dimexpr0.5\linewidth-\fboxrule-\fboxsep}
		\emph{Motivation:}
		\textit{Typüberprüfung} \\
		
		\smallskip 
		\begin{ttfamily}
			f :: (t, Char) -> (t, [Char]) \\
			f (...) = ... \\
			
			\smallskip
			
			g :: (Int, [u]) -> Int \\
			g (...) = ... \\
			
			\smallskip
			
			h = g . f
		\end{ttfamily}
	\end{minipage}
	\pause
	\begin{minipage}{\dimexpr0.5\linewidth-\fboxrule-\fboxsep}
		Beide Typausdrücke können in Übereinstimmung gebracht werden, wenn die Typterme $trans(\texttt{(t,[Char])})$ und $trans(\texttt{(Int, [u])})$ unifizierbar sind\\
		
		\smallskip
		\pause
		$\to$ \texttt{t = Int} und \texttt{u = Char}
	\end{minipage}
    \normalsize
    \end{frame}

	\begin{frame}\frametitle{Unifikationsalgorithmus}
	\normalsize
		\begin{itemize}
			\item \emph{Ziel.} Am Ende sollen nur paarweise verschiedene Variablen in der oberen Zeile stehen \underline{und} keine Regel mehr anwendbar sein.
			
			\bigskip
			
			\item \emph{beliebter Fehler.} Verwechslung von Elimination von Variablen $(x_i , x_i)$ und Dekomposition von nullären Symbolen $(\alpha , \alpha)$.
		\end{itemize}
	\end{frame}

	\begin{frame}\frametitle{Unifikationsalgorithmus -- Regeln}
	\small
		\begin{itemize}
			\item \emph{Dekomposition.} Sei $\delta \in \Sigma$ ein $k$-stelliger Konstruktor, $s_1 , \dots , s_k , t_1 , \dots , t_k$ Terme über Konstruktoren und Variablen.
			\begin{equation*}
				\begin{pmatrix}
				\delta(s_1, \dots , s_k) \\ \delta(t_1, \dots , t_k)
				\end{pmatrix}
				\quad \leadsto \quad
				\begin{pmatrix} s_1 \\ t_1 \end{pmatrix} , \dots , \begin{pmatrix} s_k \\ t_k \end{pmatrix}
			\end{equation*}
			\item \emph{Elimination.} Sei $x$ eine Variable \coloremph{!}
			\begin{equation*}
				\begin{pmatrix} x \\ x \end{pmatrix}
				\quad \leadsto \quad
				\emptyset
			\end{equation*}
			\item \emph{Vertauschung.} Sei $t$ keine Variable.
			\begin{equation*}
				\begin{pmatrix} t \\ x \end{pmatrix}
				\quad \leadsto \quad 
				\begin{pmatrix} x \\ t \end{pmatrix}
			\end{equation*}
			\item \emph{Substitution.} Sei $x$ eine Variable, $t$ keine Variable und $x$ kommt nicht in $t$ vor (occur check). Dann ersetze in jedem anderen Term die Variable $x$ durch $t$.
		\end{itemize}
	\end{frame}
	\arraycolsep2pt
	\begin{frame}\frametitle{Aufgabe 3}
		\begin{minipage}{\dimexpr0.5\linewidth-\fboxrule-\fboxsep}
		\begin{align*}
			&\left\{\left(\begin{array}{lcrlcl}
				\col{\sigma(}\sigma(&x_1,&\alpha) \col{,} &\sigma(&\gamma(x_3) , &x_3)\col{)} \\
				\col{\sigma(}\sigma( &\gamma(x_2),&\alpha) \col{,} &\sigma(&x_2 , &x_3)\col{)}
				\end{array}\right)\right\}\\
			\overset{\text{Dek.}}{\Longrightarrow}
			%
			&\left\{\left(\begin{array}{lcr}
			\col{\sigma(}&x_1\col{,}&\alpha\col{)} \\
			\col{\sigma(}&\gamma(x_2)\col{,}&\alpha\col{)}
			\end{array}\right),
			\left(\begin{array}{lcr}
			 \col{\sigma(}&\gamma(x_3) \col{,} &x_3)\col{)} \\
			 \col{\sigma(}&x_2 \col{,} &x_3)\col{)}
			\end{array}\right)\right\} \\
			\overset{\text{Dek.}}{\Longrightarrow^2}
			%
			&\left\{ \begin{pmatrix}
				x_1 \\ \gamma(x_2)
			\end{pmatrix} , \begin{pmatrix}
				\alpha \\ \alpha
			\end{pmatrix} , \begin{pmatrix}
				\gamma(x_3) \\ x_2
			\end{pmatrix} , \col{\begin{pmatrix}
				x_3 \\ x_3 
			\end{pmatrix}} \right\} \\
			\overset{\text{El.}}{\Longrightarrow}
			%
			&\left\{
			\begin{pmatrix}
				x_1 \\ \gamma(x_2)
			\end{pmatrix} , \col{\begin{pmatrix}
				\alpha \\ \alpha
			\end{pmatrix}} , \begin{pmatrix}
				\gamma(x_3) \\ x_2
			\end{pmatrix}
			\right\} \\
			\overset{\text{Dek.}}{\Longrightarrow}
			%
			&\left\{
			\begin{pmatrix}
			x_1 \\ \gamma(x_2)
			\end{pmatrix} , \col{\begin{pmatrix}
			\gamma(x_3) \\ x_2
			\end{pmatrix}}
			\right\} \\
			\overset{\text{Vert.}}{\Longrightarrow}
			%
			&\left\{
			\begin{pmatrix}
			x_1 \\ \gamma(\col{x_2})
			\end{pmatrix} , \begin{pmatrix}
			\col{x_2} \\ \gamma(x_3)
			\end{pmatrix}
			\right\} \quad {\tiny \textcolor{cdgray}{x_2 \text{ kommt nicht in } \gamma(x_3) \text{ vor}}}\\
			%
			\overset{\text{Subst.}}{\Longrightarrow}
			&\left\{
			\begin{pmatrix}
			x_1 \\ \gamma(\gamma(x_3))
			\end{pmatrix} , \begin{pmatrix}
			x_2 \\ \gamma(x_3)
			\end{pmatrix}
			\right\}
		\end{align*}
		\end{minipage}
		\pause
		\begin{minipage}{\dimexpr0.5\linewidth-\fboxrule-\fboxsep}
		\begin{center}
		\emph{allgemeinster Unifikator:} \vspace{-\baselineskip}
			\begin{align*}
				\qquad x_1 \mapsto \gamma(\gamma(x_3)) \qquad
				x_2 \mapsto \gamma(x_3) \qquad
				x_3 \mapsto x_3
			\end{align*}
			
			\medskip
			\pause
						
			\emph{weitere Unifikatoren:} \vspace{-\baselineskip}
			\begin{align*}
				x_1 &\mapsto \gamma(\gamma(\alpha))
				&x_2 &\mapsto \gamma(\alpha)
				&x_3 &\mapsto \alpha \\
				x_1 &\mapsto \gamma(\gamma(\gamma(\alpha)))
				&x_2 &\mapsto \gamma(\gamma(\alpha))
				&x_3 &\mapsto \gamma(\alpha)
			\end{align*}
			
			\medskip
			\pause
			
			\emph{Fehlschlag beim occur-check:}  \\ 
			\textcolor{cdgray}{Alphabet: $\Sigma = \left\{\gamma^{(1)} \right\}$} \vspace{-\baselineskip}
			\begin{align*}
				t_1 &= x_1 \\
				t_2 &= \gamma(x_1)
			\end{align*}
		\end{center}
		\end{minipage}
	\end{frame}

\end{document}